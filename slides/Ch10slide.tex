\documentclass{beamer}

\mode<presentation> {
	\usetheme{Pittsburgh}
}
\usepackage{mathtools}
\usepackage{mathrsfs}
\usepackage{mathtools}
\usepackage{graphicx}
\graphicspath{{./graphics/}}
\usepackage{booktabs}
\usepackage{picture}
\usepackage{algorithm}
\usepackage{algorithmic}
\usepackage{animate}
\usepackage{amsmath}
\usepackage{tikz}
\usepackage{tikz-qtree}
\usepackage{adjustbox}
\setbeamercovered{transparent}
\DeclarePairedDelimiter\abs{\lvert}{\rvert}
\hypersetup{
	colorlinks,
	linkcolor={red!50!black},
	citecolor={blue!50!black},
	urlcolor={blue!80!black}
}
\def\calA{{\cal A}}
\def\calF{{\cal F}}
\def\calP{{\cal P}}
\def\calE{{\cal E}}
\def\var{{\rm var}}

\def\bfA{{\bf A}}
\def\bfB{{\bf B}}
\def\bfC{{\bf C}}
\def\bfD{{\bf D}}
\def\bfE{{\bf E}}
\def\bfF{{\bf F}}
\def\bfG{{\bf G}}
\def\bfU{{\bf U}}
\def\bfV{{\bf V}}
\def\bfW{{\bf W}}
\def\bfX{{\bf X}}
\def\bfY{{\bf Y}}
\def\bfZ{{\bf Z}}

\def\bfa{{\bf a}}
\def\bfb{{\bf b}}
\def\bfc{{\bf c}}
\def\bfd{{\bf d}}
\def\bfe{{\bf e}}
\def\bff{{\bf f}}
\def\bfg{{\bf g}}
\def\bfu{{\bf u}}
\def\bfv{{\bf v}}
\def\bfw{{\bf w}}
\def\bfx{{\bf x}}
\def\bfy{{\bf y}}
\def\bfz{{\bf z}}

\setbeamertemplate{footline}{\insertframenumber}

\usetheme{Montpellier}  
\useoutertheme{infolines}
\usefonttheme{serif}


\title[Chapter 10]{Unsuperivised Learning} 

\author{ } 
\institute[ ]
{
	Chapter 10 \\ 
	\medskip
	\textit{ } 
}
%\date{\today}

\begin{document}
	 	
	 	\begin{frame}
	 		\titlepage % Print the title page as the first slide
	 	\end{frame}
	
 \begin{frame}
 	\frametitle{ }
 	\tableofcontents
 \end{frame}
 %###################################################################
 %###################################################################
  
 
      
      
      \begin{frame}
      	\frametitle{About this chapter}
      	\begin{itemize}
      		\item  Supervised learning aims to predicting ouput/response from features, and 
      		each training data point contains the input and output, $(\bfx_i,y_i)$. 
      		\item  Unsupervised learning do not have a specific output or response to predict, i.e. only $(\bfx_i)$. It aims at 
      		discovering the structure or characteristic of the variables in study.
      		\item  This chapter focuses on two methodologies: PCA and cluster analysis.
      		
      	 
        	\end{itemize}
        \end{frame}
        
         \begin{frame}
         	\frametitle{Unsupervised learning}
         	\begin{itemize}
         		\item Often used as part of exploratory data analysis.
         		\item Examples: 
         		     Identify the particular subgroups of stocks with close relations (clustering), 
         		     
         		     Accurate advertising based on the customers age, profession, reading, shopping habits, (clustering) ...
         		     
         		     Reduce the dimension of covariates (PCA). 
         		     
         		     ....
         		     
         		 
         		
         	\end{itemize}
         \end{frame}
        
        	
        
\section{10.1 Principal component analysis.}    
    
       \begin{frame}
       	\frametitle{ }
       	\begin{itemize}
       		\item   We have already addressed PCA in Chapter 6.
       		
       		\item Linearly combine the variables to create the new variables, called principle components.
       		
       		\item The first few explain most of the variation. 
       		
       		\item Achieve data reduction, without much loss of information.
      
       		\item Here we summarize the results and look at the examples. 
       		
       		
       		
       	\end{itemize}
       	\end{frame}
       	
       	  \begin{frame}
       	  	\frametitle{ }
       	  	\begin{itemize}
       	  		\item Data: $\bfx_i = (x_{i1}, ..., x_{ip} ) $, $i=1,...,n$.
       	  \item  Compute sample covariance matrix, e.g.
       	  ${\bf S}=\frac{1}{n}\sum_{i=1}^n (\bfx_i - \hat\mu)^T (\bfx_i - \hat\mu)$.
       	  \item Decompose into eigenvalue-eigenvector pairs:
       	  $${\bf S} = \hat {\bf e} \hat \Lambda \hat {\bf e}^T =
       	   (\hat {\bf e}_1 \vdots ... \vdots \hat {\bf e}_p) \hat \Lambda 
       	   \begin{pmatrix}
       	   \hat {\bf e}_1  \\ \vdots  \\ \hat {\bf e}_p 
       	   \end{pmatrix}$$
       	   where $\hat \Lambda = diag(\hat \lambda_1, ..., \hat \lambda_p)$. 
       	   \item $(\hat \lambda_k, \hat{\bf e}_k)$ are eigen-value-eigenvector pairs, $\hat \lambda_1 \geq ...\geq \hat \lambda_p$.
       
       	   
       		 	\end{itemize}
       		 \end{frame} 
       		
       		  \begin{frame}
       		  	\frametitle{ }
       		  	\begin{itemize}
       		  		\item  The $k$-th sample PC.s:
       		  		$$Z_k = \begin{pmatrix} z_{1k} \\ \vdots \\ z_{nk}\end{pmatrix} = {\bf X} \hat{\bf e}_k $$
       		  		 \item Component-wise, 
       		  		 $z_{ik} = x_{i1} e_{1k} + x_{i2} e_{2k} + ... + x_{ip} e_{pk}  $
       		  		 are the principle component scores of the $i$-th observation.
       		  		   \item $\hat \lambda_k$ measures the importance of the $k$-th PC.
       		  		\item  $\hat \lambda_k/(\hat \lambda_1 + ... + \hat \lambda_p)=\hat\lambda_k/trace({\bf S})$ is 
       		  		interpreted as percentage of the total variation explained by $Y_k$.
       		  		\item  Usually retain the first few PCs.
       		  		\item PCs are uncorrelated with each other.
       		  		
       		  	\end{itemize}
       		  \end{frame} 
       		  
      
     \begin{frame}
     	\frametitle{ Example: USArrests data}
     	
       For each of the
      50 states in the United States, the data set contains the number of arrests
      per 100, 000 residents for each of three crimes: Assault, Murder, and Rape.
      
      We also record UrbanPop (the percent of the population in each state living
      in urban areas). 
      
      The principal component score vectors $Z_k$ have length $n = 50$,
      and the principal component loading vectors ($\hat{\bf e}_k$) have length $p = 4$.
       
      PCA was
      performed after standardizing each variable to have mean zero and standard
      deviation one.
      
      \end{frame}
    
    \begin{frame}
    	\frametitle{ Example: USArrests data}
    \begin{center}
    	\begin{tabular}{ l   c   r }
    		\hline
    	      & PC1 & PC2  \\ \hline
    	      Murder & 0.5358995 &  −0.4181809 \\
    	      Assault  & 0.5831836 &−0.1879856\\
    	      UrbanPop &0.2781909& 0.8728062\\
    	      Rape &0.5434321 &0.1673186\\
    		\hline \\
    	\end{tabular}
    	\end{center}
     	 
     	{Table 10.1. The principal component loading vectors, $\hat{\bf e}_1$  and
     		$\hat{\bf e}_2$, for the
     		USArrests data. These are also displayed in Figure 10.1. } 
    \end{frame}   
            
    
    \begin{frame}
    	\frametitle{ }
    	\begin{figure}
    		\centering
    		\includegraphics[width=.65\linewidth]{ISLRFigures/10_1.pdf}
    		%\caption{A subfigure}
    		\caption{\scriptsize 10.1.  Next page
    		}
    	\end{figure}
    \end{frame}
    
     
     \begin{frame}
     	\frametitle{Figure 10.1 }
      
     		  Figure 10.1. The first two principal components for the USArrests data. The
     			blue state names represent the scores for the first two principal components. The
     			orange arrows indicate the first two principal component loading vectors (with
     			axes on the top and right). For example, the loading for Rape on the first component
     			is 0.54, and its loading on the second principal component 0.17 (the word
     			Rape is centered at the point (0.54, 0.17)). This figure is known as a biplot, because
     			it displays both the principal component scores and the principal component
     			loadings.
     	 
      
     \end{frame}
  
  \begin{frame}
  	\frametitle{ The  1st and 2nd PCs}
  	\begin{itemize}
  		\item  The first loading vector places approximately
  		equal weight on Assault, Murder, and Rape, with much less weight on UrbanPop
  		
  	   This component roughly corresponds to a measure of overall
  	  rates of serious crimes.
  	  
  	  \item The second loading vector places most of its weight
  	  on UrbanPop and much less weight on the other three features
  	  
  	  This
  	  component roughly corresponds to the level of urbanization of the state.
  	  
  	  \end{itemize}
  	   \end{frame}
  	   
  	   \begin{frame}
  	   	\frametitle{Discussion }
  	   	\begin{itemize}
  	  
  	  \item  The crime-related variables (Murder, Assault, and Rape)
  	  are located close to each other, and that the UrbanPop variable is far from
  	  the other three. 
  	  
  	  \item This indicates that the crime-related variables are correlated
  	  with each other---states with high murder rates tend to have high
  	  assault and rape rates----and that the UrbanPop variable is less correlated
  	  with the other three. 
  	   


 \end{itemize}
\end{frame}

\begin{frame}
	\frametitle{Discussion }
 
	
Our discussion of the loading
vectors (PC1 roughly about crime rates and PC2 about urbanization)
suggests:

States with large positive scores on the first component,
such as California, Nevada and Florida, have high crime rates;


States like North Dakota, with negative scores on the first component, have
low crime rates. 

California also has a high score on the second component,
indicating a high level of urbanization, while the opposite is true for states
like Mississippi. 

States close to zero on both components, such as Indiana,
have approximately average levels of both crime and urbanization.
10.2.2

  	
  
\end{frame} 

  
           \begin{frame}
           	\frametitle{Interpreation of PCs}
           	\begin{itemize}
           		
           		\item  Seek one direction say ${\bf b}$ with $\|\bf b\|=1$, such that
           		$$ \sum_{i=1}^n \| {\bf x}_{i} - a_i {\bf b}  \|$$
           		is the smallest. This direction is ${\bf b}  = \hat {\bf e}_1$. And $a_i = {\bf x}_i^T \bfb = <{\bf x}_i, \bfb>$
           		is the score of the $i$-th observation on the 1st PC. 
           		
           		\item  Wish to minimize rescontruction error
           		in reconstructing all 
           		 ${\bf x}_i$ using   vectors restricted to dimension $k$ linear space. 
           		 
           		 Then, the 
           		this linear space is the space spanned by $\hat {\bf e}_1, ..., \hat {\bf e}_k$, the directions of the
           		first $k$ PCs.  
           		
           	    They can be interpreted as the closest $k$ dimension linear hyperplanes to the data. 
           		
           	\end{itemize}
           \end{frame} 
            
   
         \begin{frame}
         	\frametitle{ }
         	\begin{figure}
         		\centering
         		\includegraphics[width=.45\linewidth]{ISLRFigures/10_2a.pdf}
         		 \includegraphics[width=.45\linewidth]{ISLRFigures/10_2b.pdf}
         		%\caption{A subfigure}
         		\caption{\scriptsize 10.2. Ninety observations simulated in three dimensions. Left: the
         			first two principal component directions span the plane that best fits the data. It
         			minimizes the sum of squared distances from each point to the plane. Right: the
         			first two principal component score vectors give the coordinates of the projection
         			of the 90 observations onto the plane. The variance in the plane is maximized.
         		}
         	\end{figure}
         \end{frame}
           
          
      
             
             \begin{frame}
             	\frametitle{Other issues about PCA }
             	\begin{itemize}
             		\item Scaling: 
             		\item If the variables are measuared in different units, rescale to variables to have mean 0 and variance 1 is recommended.
             		\item If the variables are of same unit and same nature (such as stock returns), PCA with both
             		rescaled and original can be conducted. 
             		\bigskip
             		  	\end{itemize}
             		  \end{frame}
             		  
             		  
             		  
             		  
             		  \begin{frame}
             		  	\frametitle{USArrest data }
             		  	\begin{itemize}
             		
             		\item In USArrest data, The four variables are measured in different unites.
       
                 Murder, Rape, and Assault are
             		reported as the number of occurrences per 100, 000 people, and UrbanPop
             		is the percentage of the state’s population that lives in an urban area.
             		These four variables have variance 18.97, 87.73, 6945.16, and 209.5
             		
             		\item Assault, a more common crime than murder and rape, have much larger variance.
             		Without standardization, it is expected to contribute much to 1st PC. 
             		
             	\end{itemize}
             \end{frame} 
         
         
         \begin{frame}
         	\frametitle{ }
         	\begin{figure}
         		\centering
         		\includegraphics[width=.65\linewidth]{ISLRFigures/10_3.pdf}
         		%\caption{A subfigure}
         		\caption{\scriptsize 10.3. Two principal component biplots for the USArrests data. Left:
         			the same as Figure 10.1, with the variables scaled to have unit standard deviations.
         			Right: principal components using unscaled data. Assault has by far the largest
         			loading on the first principal component because it has the highest variance among
         			the four variables. In general, scaling the variables to have standard deviation one
         			is recommended.
         			
         		}
         	\end{figure}
         \end{frame}
         
   
   
   \begin{frame}
   	\frametitle{  }
   	\begin{itemize}
   		\item Just like eigenvectors,   direction of a PC can be reversed.
   		
   		(${\bf e}_1$ is eighvector, so is $-{\bf e}_1$.)  
   		
   		\item Proportion of variance explained is an importance measure of the PCs.
   		
   	 
   	\end{itemize}
   \end{frame}      
         
           
            	 
            		
             
         
         \begin{frame}
         	\frametitle{ }
         	\begin{figure}
         		\centering
         		\includegraphics[width=.7\linewidth]{ISLRFigures/10_4.pdf}
         		%\caption{A subfigure}
         		\caption{\scriptsize 10.4.  Left: a scree plot depicting the proportion of variance explained
         			by each of the four principal components in the USArrests data. Right: the cumulative
         			proportion of variance explained by the four principal components in the
         			USArrests data.
         		
         		}
         	\end{figure}
         \end{frame}
         
          \begin{frame}
          	\frametitle{  Number of PCs to retain}
          	\begin{itemize}
          		\item  No universal rule.
          		\item  Problem specific.
          		 \item Recommended:
          		 \begin{enumerate}
          		 \item Look for elbow in the scree plot.
          			\item Enough PCs to explain 90\% of total variation.
          			\item PCs with variance larger than average.
				\item Horn's Parallel Analysis with random permutations (Random Matrix Theory)
          		\end{enumerate}
          		\item This is useful to regression, classification and cluster analysis 
          		to work with the first few PCs rather than all the $p$ 
          		inputs. $p$ could   be  too large and contain many noisy or useless inputs.
          	 
          		
          	\end{itemize}
          \end{frame}    
                  
 \section{10.2. Clustering methods}
 
   \begin{frame}
   	\frametitle{ About cluster analysis}
   	\begin{itemize}
   		
   		\item 
   		Techniques for finding subgroups or data points, or
   		clusters, in a data set, so that the observations within
   		each group are quite similar to each other. 
   		\item An unsupervised problem:   to discover
   		structure --- in this case, distinct clusters --- on the basis of a data set.
   		 \item 
   		Both clustering and PCA seek to simplify the data via a small number
   		of summaries, but their mechanisms are different:
   		\begin{enumerate}
   			\item PCA looks to find a low-dimensional representation of the observations
   			that explain a good fraction of the variance;
   		\item Clustering looks to find homogeneous subgroups among the observations.
   		 
   		\end{enumerate}
   		
   		  	\end{itemize}
   		  \end{frame}    
   		  
   		 
   		  
   		  \begin{frame}
   		  	\frametitle{ Market segmentation}
   		  	\begin{itemize}
   		 \item The goal is to perform market segmentation by identifying
   		subgroups of people who might be more receptive to a particular form
   		of advertising, or more likely to purchase a particular product. 
   		
   		\item The task of
   		performing market segmentation amounts to clustering the people in the
   		data set.
   	      		
   	      	\end{itemize}
   	      \end{frame}    
   	      
   	      
   	      
   	      
   	      \begin{frame}
   	      	\frametitle{ K-means clustering and hierarchical clustering }
   	      	\begin{itemize}
   	      		
   	   \item  Two best-known
   		clustering approaches: K-means clustering and hierarchical clustering.  
   		\begin{enumerate}
   		\item K-means clustering:  seek to partition the observations into a pre-specified
   		number of clusters. 
   		\item Hierarchical clustering: 
   		a tree-like visual representation of the observations, called a {\color{red} dendrogram},
   		that allows us to view at once the clusterings obtained for each possible
   		number of clusters, from 1 to n. 
   		\end{enumerate}
   		 \item
   		Cluster observations on the basis of the features in
   		order to identify subgroups among the observations;
   		
   		Or cluster features
   		on the basis of the observations in order to discover subgroups
   		
     	\end{itemize}
     \end{frame}    
     
    
     
     
     \begin{frame}
     	\frametitle{K-Means clustering}
     	\begin{itemize}
     		\item Partition the data set of $n$ observations into $K$ distinct, non-overlapping subsets.
     		
     		Each set, denoted as $C_k$, $k=1, .., K$,  is called a cluster.
     		
     		\item Good clustering:  the
     		within-cluster variation is as small as possible
     		
     		\item Let $W(C_k)$ be a measure of the within-cluster variation for cluster $C_k$.
     		
     		\item We wish to minimize the total within-cluster variations 
     		
     		$$ \hbox{minmize}_{C_1, ..., C_K} \Bigl\{ \sum_{i=1}^K W(C_K)\}$$
     		
     			\end{itemize}
     		\end{frame} 
     		
     	
     	\begin{frame}
     		\frametitle{K-Means clustering}
     		\begin{itemize}
     			\item Several different ways to define $W(C_k)$. 
     			
     			\item Using squared Euclidan distance, we define
     			$$ W(C_k)= {1 \over |C_k|}   \sum_{i, j \in C_k} \|\bfx_i - \bfx_j\|^2$$
     			 
     		 
     
     		 
     		\end{itemize}
     	\end{frame} 	
         
         \begin{frame}
         	\frametitle{ }
         	\begin{figure}
         		\centering
         		\includegraphics[width=.8\linewidth]{ISLRFigures/10_5.pdf}
         		%\caption{A subfigure}
         		\caption{\scriptsize   10.5. A simulated data set with 150 observations in two-dimensional
         			space. Panels show the results of applying K-means clustering with different values
         			of K, the number of clusters. The color of each observation indicates the cluster
         			to which it was assigned using the K-means clustering algorithm. Note that
         			there is no ordering of the clusters, so the cluster coloring is arbitrary. These
         			cluster labels were not used in clustering; instead, they are the outputs of the
         			clustering procedure.
         		}
         	\end{figure}
         \end{frame}
         
        
         
         \begin{frame}
         	\frametitle{K-Means cluster algorithm}
         	
         	 Algorithm 10.1 K-Means Clustering
         	\begin{itemize}
         		\item
         		1. Randomly assign a number, from 1 to K, to each of the observations.
         		These serve as initial cluster assignments for the observations.
         	\item 	2. Iterate until the cluster assignments stop changing:
         	\begin{enumerate}
         		\item   For each of the K clusters, compute the cluster centroid. The
         		kth cluster centroid is the vector of the p feature means for the
         		observations in the kth cluster.
         		\item Assign each observation to the cluster whose centroid is closest
         		(where closest is defined using Euclidean distance).
         	\end{enumerate}	
         		
         		
         		
         	\end{itemize}
         \end{frame} 	
         
          
           \begin{frame}
           	\frametitle{K-Means cluster algorithm}
           	\begin{itemize}
           		\item The objective function always decreases at each step.
           		
           		\item $${1 \over |C_k|} \sum_{i, j\in C_k} \|\bfx_i - \bfx_j\|^2
           	 = 2 \sum_{i \in C_k} \|\bfx_i - \bar \bfx\|^2
           	 $$
           	 where $\bar \bfx$ is the sample mean $\bfx_i$ for $i \in C_k$   
      
           		\item $K$-means algorithm finds a local minimum. 
           		
           		\item The result depends on the initial (random) cluster assignment.
           		
           		\item Try several different initials, and select the best result (the smallest 
           		objective function).
           	\end{itemize}
           \end{frame} 	
           
                \begin{frame}
                	 
                	\begin{figure}
                		\centering
                		
                		\centering
                		\includegraphics[width=.65\linewidth]{ISLRFigures/10_6.pdf}
                		%\caption{A subfigure}
                		\caption{10.6
                		}
                		
                		
                	\end{figure}
                \end{frame}
                
             
              
                \begin{frame}
                 
                	 FIGURE 10.6. The progress of the K-means algorithm on the example of Figure
                	 10.5 with $K=3$. Top left: the observations are shown. Top center: in Step 1
                	 of the algorithm, each observation is randomly assigned to a cluster. Top right:
                	 in Step 2(a), the cluster centroids are computed. These are shown as large colored
                	 disks. Initially the centroids are almost completely overlapping because the
                	 initial cluster assignments were chosen at random. Bottom left: in Step 2(b),
                	 each observation is assigned to the nearest centroid. Bottom center: Step 2(a) is
                	 once again performed, leading to new cluster centroids. Bottom right: the results
                	 obtained after ten iterations.
                		 		
                		  
                		 \end{frame} 
                		\begin{frame}
                		 
                			\begin{figure}
                				\centering
                				
                				\centering
                				\includegraphics[width=.65\linewidth]{ISLRFigures/10_7.pdf}
                				%\caption{A subfigure}
                				\caption{10.7
                				}
                				
                				
                			\end{figure}
                		\end{frame} 
                		 
                		 
                		  \begin{frame}
                		  	 
                		   
                		  		FIGURE 10.7. K-means clustering performed six times on the data from Figure
                		  		10.5 with $K = 3$, each time with a different random assignment of the observations
                		  		in Step 1 of the K-means algorithm. Above each plot is the value of
                		  		the objective (10.11). Three different local optima were obtained, one of which
                		  		resulted in a smaller value of the objective and provides better separation between
                		  		the clusters. Those labeled in red all achieved the same best solution, with an
                		  		objective value of 235.8.
                		  
                
                \end{frame} 
                
                  
                \begin{frame}
                	\frametitle{Hierarchical  clustering }
                	\begin{itemize}
                		\item  $K$-means clustering requres pre-specified number of clusters, a disadvantage.
                		 \item Hierarchical clustering   does not require that.
                		 \item It results in a tree-based representation of the
                		 observations, called a dendrogram.
                		 \item {\it  bottom-up } or {\it  agglomerative }
                		  clustering
                		 
                		
                	\end{itemize}
                \end{frame}   
                
              
                
                \begin{frame}
                	
                	\begin{figure}
                		\centering
                		
                		\centering
                		\includegraphics[width=.55\linewidth]{ISLRFigures/10_8.pdf}
                		%\caption{A subfigure}
                		\caption{\scriptsize 10.8. Forty-five observations generated in two-dimensional space. In
                			reality there are three distinct classes, shown in separate colors. However, we will
                			treat these class labels as unknown and will seek to cluster the observations in
                			order to discover the classes from the data.
                		}
                		
                		
                	\end{figure}
                \end{frame} 
                
                \begin{frame}
                	\begin{figure}
                		\centering
                		
                		\centering
                		\includegraphics[width=.8\linewidth]{ISLRFigures/10_9.pdf}
                		%\caption{A subfigure}
                		\caption{\tiny 10.9. Left: dendrogram obtained from hierarchically clustering the data
                			from Figure 10.8 with complete linkage and Euclidean distance. Center: the dendrogram
                			from the left-hand panel, cut at a height of nine (indicated by the dashed
                			line). This cut results in two distinct clusters, shown in different colors. Right:
                			the dendrogram from the left-hand panel, now cut at a height of five. This cut
                			results in three distinct clusters, shown in different colors. Note that the colors
                			were not used in clustering, but are simply used for display purposes in this figure.
                		}
                		
                		
                	\end{figure}
                \end{frame} 
                
                  
                 \begin{frame}
                 	\frametitle{Interpreting a dendrogram }
                 	\begin{itemize}
                 		\item Each leaf of the dendrogram represents
                 		one of the 45 observations in Figure 10.8. 
                 		
                 		\item However, as we move
                 		up the tree, some leaves begin to fuse into branches. These correspond to
                 		observations that are similar to each other.
                 		
                 		\item As we move higher up the tree,
                 		branches themselves fuse, either with leaves or other branches.
                 		
                 		        
                 		 \item The earlier
                 		 (lower in the tree) fusions occur, the more similar the groups of observations
                 		 are to each other.
                 		 
                 		\item Observations that fuse later
                 		(near the top of the tree) can be quite different
                 	 
                 	\end{itemize}
                 \end{frame}   
              
                 \begin{frame}
                 	\frametitle{A rough closeness measure}
                 	\begin{itemize}
                 		\item  For any two observations, we can look for the point in
                 		the tree where branches containing those two observations are first fused.
                 		The height of this fusion, as measured on the vertical axis, indicates how
                 		different the two observations are
                 		\item Observations that fuse at the very
                 		bottom of the tree are quite similar to each other, whereas observations
                 		that fuse close to the top of the tree will tend to be quite different. 
                 		
                 	 
                 		
                 	 
                 	\end{itemize}
                 \end{frame}   
                 
                           
                           
                           \begin{frame}
                           	\frametitle{Interpreting dendrogram}
                           	\begin{figure}
                           		\centering
                           		
                           		\centering
                           		\includegraphics[width=.7\linewidth]{ISLRFigures/10_10.pdf}
                           		%\caption{A subfigure}
                           		\caption{\scriptsize 10.10. An illustration of how to properly interpret a dendrogram with
                           			nine observations in two-dimensional space. Left: a dendrogram generated using
                           			Euclidean distance and complete linkage. Observations 5 and 7 are quite similar
                           			to each other, as are observations 1 and 6. However, observation 9 is no more
                           			similar to observation 2 than it is to observations 8, 5, and 7, even though observations
                           			9 and 2 are close together in terms of horizontal distance. This is because
                           			observations 2, 8, 5, and 7 all fuse with observation 9 at the same height, approximately
                           			1.8. Right: the raw data used to generate the dendrogram can be used to
                           			confirm that indeed, observation 9 is no more similar to observation 2 than it is
                           			to observations 8, 5, and 7.
                           			Now
                           		}
                           		
                           		
                           	\end{figure}
                           \end{frame} 
                           
                           
                       	       \begin{frame}
                       	       	\frametitle{ Identifying clusters}
                       	       	\begin{itemize}
                       	       		\item Make a horizontal cut across the
                       	       		dendrogram, as   Figure 10.9
                       	       		
                       	       		\item  The distinct sets of observations beneath the cut can be interpreted as clusters.
                       	       		
                       	       	    \item The lower cuts create more clusters. The higher cuts create less clusters.
                       	       	    
                       	       	    \item One single dendrogram can be used to obtain any number of
                       	       	    clusters.
                       	       	    
                       	       		\item  Choice of cuts can even be done by visual judgment of the dendrogram.
                       	       	 
                       	       		\item When hierarchical structure does not exist in data, the hirarchical clustering could be worse than K-means clustering.
                       	       		
                       	       	\end{itemize}
                       	       \end{frame}   
               
          
              
                
   
  
               
                   \begin{frame}
                   	\frametitle{ Hierarchical clustering algorithm}
                   	\begin{itemize}
                   		\item  
                   		1. Begin with $n$ observations and a measure (such as Euclidean distance)
                   		of all the
                   	  ${n \choose 2} = n(n-1)/2   $ pairwise dissimilarities. Treat each
                   		observation as its own cluster.
                   	\item 	2. For $i=n, n-1, ...2$:
                   	\begin{enumerate} 
                   		\item 
                   		  Examine all pairwise inter-cluster dissimilarities among the i
                   		clusters and identify the pair of clusters that are least dissimilar
                   		(that is, most similar). Fuse these two clusters. The dissimilarity
                   		between these two clusters indicates the height in the dendrogram
                   		at which the fusion should be placed.
                   		\item  Compute the new pairwise inter-cluster dissimilarities among
                   		the $i -1$ remaining clusters.
                   		 
                   	\end{enumerate}	 
                   		
      
                   		
                   	\end{itemize}
                   \end{frame}   
                   
                  
             
                   \begin{frame}
                   	\frametitle{ {\it Linkage}: the dissimilarity measure between two clusters }
                   	\centering 
                   	{\scriptsize  
                   	\begin{tabular}{ | l |  l   |}
                   		\hline
                   	Linkage &  Description \\
                   	\hline 
                   	Complete & 
                   	Maximal intercluster dissimilarity. Compute all pairwise\\
                   	&  dissimilarities between the observations in cluster A and the
                    \\ & 	   observations   in cluster B, and record the largest of these  
                   	dissimilarities. \\
                   	
                   	\hline
                   	Single & 
                   	Minimal intercluster dissimilarity. Compute all pairwise
                   	 \\ &   dissimilarities between the observations in cluster A and the
                   	observations
                   	\\ &   in cluster B,  and record the smallest of these
                   	dissimilarities. Single 
                   	 \\ &  linkage can result in extended, trailing
                   	clusters in which single \\ & observations are fused one-at-a-time. \\
                   	  \hline
                   	Average & 
                   	Mean intercluster dissimilarity. Compute all pairwise dissimilarities
                   	 \\ & between the observations in cluster A and the
                   	observations in cluster B, 
                   	 \\ & and record the average of these
                   	dissimilarities. \\
  \hline
                   	Centroid & 
                   	Dissimilarity between the centroid for cluster A (a mean
                   	vector \\ &   of length $p$) and
                   	  the centroid for cluster B. Centroid
                   	linkage can \\ &   result in undesirable inversions.
                   	 \\  
                   		\hline
                   	\end{tabular}
                   }
                   
                 {\scriptsize   TABLE 10.2. A summary of the four most commonly-used types of linkage}
                   \end{frame}
                   
                    
              \begin{frame}
              \frametitle{ }
              \begin{figure}
              	\centering
              	\includegraphics[width=.5\linewidth]{ISLRFigures/10_11.pdf}
              	%\caption{A subfigure}
              	\caption{\scriptsize 10.11. An illustration of the first few steps of the hierarchical
              		clustering algorithm, using the data from Figure 10.10, with complete linkage
              		and Euclidean distance. Top Left: initially, there are nine distinct clusters,
              		$\{1\}, \{2\}, . . . , \{9\}$. Top Right: the two clusters that are closest together, $\{5\}$ and
              		$\{7\}$, are fused into a single cluster. Bottom Left: the two clusters that are closest
              		together, $\{6\}$ and $\{1\}$, are fused into a single cluster. Bottom Right: the two clusters
              		that are closest together using complete linkage, $\{8\}$ and the cluster $\{5, 7\}$,
              		are fused into a single cluster.
              		 
              	}
              \end{figure}
            \end{frame}
                  
                   \begin{frame}
                   	\frametitle{ }
                   	\begin{figure}
                   		\centering
                   		\includegraphics[width=.6\linewidth]{ISLRFigures/10_12.pdf}
                   		%\caption{A subfigure}
                   		\caption{\scriptsize 10.12. Average, complete, and single linkage applied to an example
                   			data set. Average and complete linkage tend to yield more balanced clusters.
                   		}
                   	\end{figure}
                   \end{frame}
                   
                 
                    \begin{frame}
                    	\frametitle{ Dissimilarity measure }
                    	\begin{itemize}
                    		\item  Very important and can greatly affect the final result.
                    		
                    		\item  Euclidean distrance.
                    		
                    		\item  Correlation based distrance: if two observations have high correlation, the distance is closer. 
                    		
                    		(Caution: this is not correlation between two variables, but between two observations.)
                    		
                    		\item   Different problem may need different dissimilarity measure. 
                    		
                    	\end{itemize}
                    \end{frame}   
                   
            
            \begin{frame}
            	\frametitle{ The online shopping example}
            	\begin{itemize}
            		\item Using Euclidean distance may not be appropriate.
            		
            		Those with same shopping habit but different shopping volume should be but may not be 
            		clustered together.
            	 
            	\item  Using correlation based distrance is more appropriate. 
            	
            	\item   Variable scaling, as in PCA, whether the variables should be standardized is problem specific.
            	
            	 Example:  High-frequency purchases like socks therefore tend
            	to have a much larger effect on the inter-shopper dissimilarities, and hence
            	on the clustering ultimately obtained, than rare purchases like computers.
            	This may not be desirable.
            	
            	\item Variables measured in different units should be standardized.
             	
            	\end{itemize}
            \end{frame}   
            
            \begin{frame}
            	\frametitle{ }
            	\begin{figure}
            		\centering
            		\includegraphics[width=.7\linewidth]{ISLRFigures/10_14.pdf}
            		%\caption{A subfigure}
            	 
            	\end{figure}
            \end{frame}
            
               \begin{frame}
               	\frametitle{}
              {\scriptsize  
               	  FIGURE 10.14. An  online retailer sells two items: socks and computers.
               	  Left: the number of pairs of socks, and computers, purchased by eight online shoppers
               	  is displayed. Each shopper is shown in a different color. If inter-observation
               	  dissimilarities are computed using Euclidean distance on the raw variables, then
               	  the number of socks purchased by an individual will drive the dissimilarities obtained,
               	  and the number of computers purchased will have little effect. This might be
               	  undesirable, since (1) computers are more expensive than socks and so the online
               	  retailer may be more interested in encouraging shoppers to buy computers than
               	  socks, and (2) a large difference in the number of socks purchased by two shoppers
               	  may be less informative about the shoppers’ overall shopping preferences than a
               	  small difference in the number of computers purchased. Center: the same data
               	  is shown, after scaling each variable by its standard deviation. Now the number
               	  of computers purchased will have a much greater effect on the inter-observation
               	  dissimilarities obtained. Right: the same data are displayed, but now the y-axis
               	  represents the number of dollars spent by each online shopper on socks and on
               	  computers. Since computers are much more expensive than socks, now computer
               	  purchase history will drive the inter-observation dissimilarities obtained.
               	}
               	   
               		  \end{frame} 
               		  
               		 \begin{frame}
               		 	\frametitle{}
               		 	\begin{itemize}
               		\item 
               		  Should the observations or features first be standardized in some way?
               		For instance, maybe the variables should be centered to have mean
               		zero and scaled to have standard deviation one.
               	
               	\item In the case of hierarchical clustering,
               	
               		– What dissimilarity measure should be used?
               		
               		– What type of linkage should be used?
               		
               		– Where should we cut the dendrogram in order to obtain clusters?
               	\item  In the case of K-means clustering, how many clusters should we look
               		for in the data?
               	
               		
               	\end{itemize}
               \end{frame}  
                
                
             \begin{frame}
             	\frametitle{ }
             	\begin{itemize}
             		\item Forcing every observation, including outliers, into clusters, can distort
             		the final outcome.
        (A soft version  of $K$-means clustering by mixture model may help)
             		
             		\item Clustering methods generally are not very robust to perturbations
             		to the data.
             		
             		\item Performing clustering with
             		different choices of these parameters (linkage, standardization or not, etc), and looking at the full set of results
             		\item Clustering subsets of the data in order to get a
             		sense of the robustness
             	
             		
             	\end{itemize}
             \end{frame}  
           
                       
               
               
 
                      
      		\begin{frame}
      			\frametitle{Exercises  }
      			
      			
      			{\sl  Run the R-Lab codes in Section 10.4 of ISLR
      				
      				Exercises
      	 1-3  and 10 of Section  10.7 of ISLR 
      			}
      			
   
      		\end{frame}
      		
      			\begin{frame}
      				\frametitle{  }
      				
      				
      				
      				End of Chapter 10. 
      				
      				
      			\end{frame}
      
    \end{document}
    
    
 