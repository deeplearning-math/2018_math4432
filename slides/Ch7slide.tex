\documentclass{beamer}

\mode<presentation> {
	\usetheme{Pittsburgh}
}
\usepackage{mathtools}
\usepackage{mathrsfs}
\usepackage{mathtools}
\usepackage{graphicx}
\graphicspath{{./graphics/}}
\usepackage{booktabs}
\usepackage{picture}
\usepackage{algorithm}
\usepackage{algorithmic}
\usepackage{animate}
\usepackage{amsmath}
\usepackage{tikz}
\usepackage{tikz-qtree}
\usepackage{adjustbox}
\setbeamercovered{transparent}
\DeclarePairedDelimiter\abs{\lvert}{\rvert}
\hypersetup{
	colorlinks,
	linkcolor={red!50!black},
	citecolor={blue!50!black},
	urlcolor={blue!80!black}
}
\def\calA{{\cal A}}
\def\calF{{\cal F}}
\def\calP{{\cal P}}
\def\calE{{\cal E}}
\def\var{{\rm var}}

\def\bfA{{\bf A}}
\def\bfB{{\bf B}}
\def\bfC{{\bf C}}
\def\bfD{{\bf D}}
\def\bfE{{\bf E}}
\def\bfF{{\bf F}}
\def\bfG{{\bf G}}
\def\bfU{{\bf U}}
\def\bfV{{\bf V}}
\def\bfW{{\bf W}}
\def\bfX{{\bf X}}
\def\bfY{{\bf Y}}
\def\bfZ{{\bf Z}}

\def\bfa{{\bf a}}
\def\bfb{{\bf b}}
\def\bfc{{\bf c}}
\def\bfd{{\bf d}}
\def\bfe{{\bf e}}
\def\bff{{\bf f}}
\def\bfg{{\bf g}}
\def\bfu{{\bf u}}
\def\bfv{{\bf v}}
\def\bfw{{\bf w}}
\def\bfx{{\bf x}}
\def\bfy{{\bf y}}
\def\bfz{{\bf z}}
\def \bfbeta {\boldsymbol{\beta}}
\def \bfalpha {\boldsymbol{\alpha}}

\setbeamertemplate{footline}{\insertframenumber}

\usetheme{Montpellier}  
\useoutertheme{infolines}
\usefonttheme{serif}


\title[Chapter 7]{Moving Beyond Linearity} 

\author{Yuan YAO} 
\institute[HKUST]
{
	Chapter 7 \\ 
	\medskip
	\textit{ } 
}
%\date{\today}

\begin{document}
	 	
	 	\begin{frame}
	 		\titlepage % Print the title page as the first slide
	 	\end{frame}
	
 \begin{frame}
 	\frametitle{ }
 	\tableofcontents
 \end{frame}
 %###################################################################
 %###################################################################
           
      
      \begin{frame}
      	\frametitle{About this chapter}
      	\begin{itemize}
      		\item  Linear model is the most fundamental statistical model.
      		\item  Its limitation is the mean response must be a linear function of inputs/covariates.
      		\item  This relation in practice often does not hold.
      		\item  Nonlinear models are needed  
        	\end{itemize}
        \end{frame}
        
       
       \begin{frame}
       	\frametitle{The nonlinear models. }
       	\begin{itemize}
       		\item Polynomial regression.
       		\item Step functions 
       		\item Regression splines
       		\item Smoothing splines
       		\item Local regression
       		\item Generalized additive models.
       		\item Trees, SVM, neural nets, ...	
       	\end{itemize}
       \end{frame} 
       
        
       
\section{Polynomial regression}
        
         \begin{frame}
         	\frametitle{Some general remark}
         	\begin{itemize}
         	
         		\item  Rather than directly using inputs, we use polynomials, or step functions, of the inputs 
         		as the ``derived inputs", in linear regression.
         		\item  The approach can be viewed  as derived inputs approach.
         		\item More generally, the basis function approach. 
         	  \item Starting from now, we only consider one input, for simplicty of illustration.
         	\end{itemize}
         \end{frame} 
        
          \begin{frame}
          	\frametitle{}
          	\begin{itemize}
          			\item  Data: $(y_i, x_i), i=1,...,n$.
          		\item  The genearl model
          		$$ y_i = f(x_i) + \epsilon_i$$
          		\item  We can safely assume $f(\cdot)$ to be continuous.
          		\item  Cannot search for arbitrary function $f(\cdot)$.
          		\item Limit the search space.
          		\item Continuous functions? (still infinite dimension but can be approxiamted.
          		\item by Polynomial functions, or step functions, or certain basis functions,...
          			\end{itemize}
          		\end{frame} 
          		
          		\begin{frame}
          			\frametitle{}
          			\begin{itemize}
          		 
          		\item  Linear model (retricting $f(\cdot)$ to be linear) :
          		$$ y_i = \beta_0 + \beta_1 x_i + \epsilon_i$$
          		\item  Polynomial regression model (restricting $f(\cdot)$ to be polynomial of degree $p$):
          			$$ y_i = \beta_0 + \beta_1 x_i + \beta_2 x_i^2 + ...+ \beta_p x_i^p + \epsilon_i$$
          		\item This is a multiple linear regression model with $p$ inputs:
          		$(x_i, x_i^2, ..., x_i^p)$.
          		\item All linear regression results apply.
          		\item Problem: how to determine the appropriate degree $p$.
          	   \item Drawback: difficult to fit locally highly varying functions.
          	\end{itemize}
          \end{frame} 
       
         \begin{frame}
         	\frametitle{The generalized  linear model}
         	\begin{itemize}
         		
         		\item  Generalized linear model:
         		$$ E(Y|X) = g(X^T\beta)$$
         		where $g$ is a given {\it link function}
         		\item  Examples:
         		\begin{enumerate}
         			\item linear regression: $g(x)= x$
         			\item logistic regression: $g(x)= 1/(1+ e^{-x})$, the sigmoid function. $Y=1$ or $0$.
         			\item Probit model: $g(x)= \Phi(x)$, the cdf of $N(0, 1)$. $Y=1$ or $0$.
         			\item Poisson model: $g(x) = e^x$. $Y$ is count data.
         			\item ...
         		\end{enumerate}
         	 
         		\item They can be extended to generalized non-linear model in the same fashion.
         		
         	\end{itemize}
         \end{frame} 
         
           \begin{frame}
           	\frametitle{ logistic model with polynomial regression}
           	\begin{itemize}
           		
           		\item  For binary response $y_i$, coded the binary events as $1$ and $0$. 
           		$$ p(y_i =1|x_i) =  \frac{  \exp(\beta_0 + \beta_1 x_i + ...+ \beta_p x_i^p) }{ 1 + \exp(\beta_0 + \beta_1 x_i + ...+ \beta_p x_i^p)} $$
           		 
           		\item   This is essentially just logistic model with $p$ inputs.
           	 
           		\item All results on logistic model apply here.	
           	\end{itemize}
           \end{frame} 
        
         
           
           \begin{frame}
           	\frametitle{ }
           	\begin{figure}
           		\centering
           		 
           			\centering
           			\includegraphics[width=.8\linewidth]{ISLRFigures/7_1.pdf}
           			%\caption{A subfigure}
           		\caption{\scriptsize 7.1. The Wage data. Left: The solid blue curve is a degree-4 polynomial
           			of wage (in thousands of dollars) as a function of age, fit by least squares. The
           			dotted curves indicate an estimated 95\% confidence interval. Right: We model the
           			binary event wage$>250$ using logistic regression, again with a degree-4 polynomial.
           			The fitted posterior probability of wage exceeding \$250,000 is shown in blue, along
           			with an estimated 95\% confidence interval.
           		}
           	\end{figure}
           \end{frame}
           
\section{Step functions}

  \begin{frame}
    \frametitle{ Step functions (piecewise constant functions)}
    \begin{itemize}
       \item Step functions are piece-wise constants.
       \item Continuous functions can be well approximated by step functions.
       A function can be approximated by step functions.
        \item  Create the cutpoints 
        $$-\infty = c_0 < c_1< ...< c_p < c_{p+1}=\infty$$  
       \item The entire real line is cut into $p+1$ intervals.
       \item Set $c_k(x) = I(c_k \leq x < c_{k+1})$, for $k=0, ..., p$.  
       \item Use linear combination of $c_k(x)$ to approximate functions.	
              		
    \end{itemize}
  \end{frame} 
  
  \begin{frame}
  	\frametitle{ Regression model based on step functions}
  	\begin{itemize}
  		\item 
  		$$ y_i = \beta_0 + \beta_1 c_1(x_i) + ... + \beta_p c_p(x_i) + \epsilon_i.$$
  		
  		\item Again a multiple linear regression model. 
  		\item Same extension works for generalized linear model.
  		\item Difficulty in creating the number and locations of cutpoints 
  		\item Drawback: non-smooth, not even continuous.
  		
  	\end{itemize}
  \end{frame} 
  
            
             \begin{frame}
             	\frametitle{ }
             	\begin{figure}
             		\centering
             		
             		\centering
             		\includegraphics[width=.7\linewidth]{ISLRFigures/7_2.pdf}
             		%\caption{A subfigure}
             		\caption{7.2. The Wage data. Left: The solid curve displays the fitted value from
             			a least squares regression of wage (in thousands of dollars) using step functions
             			of age. The dotted curves indicate an estimated 95\% confidence interval. Right:
             			We model the binary event wage$>250$ using logistic regression, again using step
             			functions of age. The fitted posterior probability of wage exceeding \$250,000 is
             			shown, along with an estimated 95\% confidence interval.
             		}
             	\end{figure}
             \end{frame}
             
              
              \begin{frame}
              	\frametitle{ Basis functions}
              	\begin{itemize}
              		\item In general, let $b_1(x),..., b_p(x)$ be a set of  {\it  basis functions.}
              		\item We limit the search space of $f(\cdot)$ to the space that is linearly spanned by
              		these basis functions:
              		$$ \{ g(x): g(x)= a_0 +  \sum_{j=1}^p a_i b_i(x) \}$$
              		\item The model is 
              		$$ y_i = \beta_0 + \beta_1 b_1(x_i) + ... + \beta_p b_p(x_i) + \epsilon_i.$$
              		
              		\item Again a multiple linear regression model. 
              		\item The polynomial functions or step functions are special cases of basis functions approach.
              		\item Other choices: wavelet functions or Fourier series or regression splines.
              	\end{itemize}
              \end{frame} 
\section{Regression splines}

  \begin{frame}
  	\frametitle{ Piecewise polynomial functions}
  	\begin{itemize}
  		\item A hybrid of step function approach and polynomal function approach.
  		\item Cut the entire real line (or the range of values of covariates) into sub-intervals same as
  		step function approach.
  		\item These cutpoints are called knots. 
  		\item Use a polynomial function   on each sub-interval.
  		\item Still a multiple linear regression model.
  		
  		\item Step function approach is a special case of piecewise polynomial of degree $0$.
  		
  		\item Advantage: capture local variation; the degree of polynomial is generally low.
  		\item disadvantage: dis-continuity at knots.
  	 
  	\end{itemize}
  \end{frame} 
  
               \begin{frame}
               	\frametitle{ }
               	\begin{figure}
               		\centering
               		
               		\centering
               		\includegraphics[width=.8\linewidth]{ISLRFigures/7_3.pdf}
               		%\caption{A subfigure}
               		
               	\end{figure}
               \end{frame}
               
                \begin{frame}
                	
               Figure 7.3. (Figure of previous page) Various piecewise polynomials are fit to a subset of the Wage
               	data, with a knot at age$=50$. Top Left: The cubic polynomials are unconstrained.
               	Top Right: The cubic polynomials are constrained to be continuous at age$=50$.
               	Bottom Left: The cubic polynomials are constrained to be continuous, and to
               	have continuous first and second derivatives. Bottom Right: A linear spline is
               	shown, which is constrained to be continuous
              
               \end{frame} 
               
               \begin{frame}
               	\frametitle{Constraining the piecewise polynomial}
               	\begin{itemize}
               		\item When fit the least squares, one can add constriants to 
               		the least squares minimization  
               		\item The constraints can be such that the piecewise polynomial is forced 
               		to be continuous at knots.
               		\item The constraints can be stronger such that the piecewise polynomial is forced 
               		to be differentiale at knots with continuous first derivatives. 
               		
               		\item The constraints can be stronger such that the piecewise polynomial is forced 
               		to be differentiale at knots with continuous second derivatives.
               		
               		 \item ...
               		
               	\end{itemize}
               \end{frame} 
               
                 \begin{frame}
                 	\frametitle{The effect of constraints  }
                 	\begin{itemize}
                 		\item Each constraint can be expressed as on linear equation. 
                 		\item It reduces one degree of freedom.
                 		\item And reduces the complexity of the model. 
                 		
                 	  
                 	\end{itemize}
                 \end{frame} 
                 
                  
                  \begin{frame}
                  	\frametitle{Spline functions  }
                  	\begin{itemize}
                  		\item Spline functions of degree $d$ are piecewise polynomial functions
                  		of degree $d$ but have continuous derivatives up to order $d-1$ at knots.
                  		
                  		\item Cubic spline: piecewise cubic polynomials but are continuous and have continous 1st and
                  		second derivatives at knots.
                  		
                  		\item The degree of freedom of a cubic spline with $K$ knots is:
                  		$$ 4\times (K+1) - 3K = K+4.$$
                  		Totally $K+1$ cubic functions, each has 4 free parameters, but
                  		each of the $K$ knot has 3 constraints on continuity, continuity of 1st and 2nd derivatives. 
                  		
                  	\end{itemize}
                  \end{frame} 
                  
                   \begin{frame}
                   \frametitle{Spline basis representation   }
                   \begin{itemize}
                   \item  Suppose the $K$ knots $\xi_1< ... < \xi_K$ are determined.
                   \item We may find $1, b_1(x), ..., b_{K+3}$ to form the space of cubic splines with knots
                   at $\xi_1, ..., \xi_K$. 
                   \item Then the spline regression model is 
                   $$y_i = \beta_0 + \eta_1 b_1(x_i) + ... + \beta_K b_{K+3}(x_i) + \epsilon_i$$
                   \item How to find these basis functions $b_k(x)$?
                   \item Each must be a polynomial of order 3 and must be continuous, continous at
                   1st and 2nd derivates at all knots.
                  
                   \end{itemize}
                   \end{frame} 
                  
              
                   \begin{frame}
                   	\frametitle{Spline basis representation   }
                   	\begin{itemize}
                   		\item $x $ , $x^2$ and $x^3$ satisfy the requirement.
                   		\item Let 
                   		$$ h(x, \xi)= (x-\xi)_+^3= \begin{cases} 
                   		(x-\xi)^3 & \text{if  } x > \xi   \\
                   		0   & \text{otherwise} 
                   		\end{cases}$$
                   		
                   		\item $h(x, \xi_k)$ also satisfy the requirement.
                   		\item The basis functions of cubic splines can be
                   		$$1, x, x^2, x^3, h(x, \xi_1), ..., h(x, \xi_K)$$
                   		\item Totally   $K+4$ dimension with $K+3$ features.
                   		
                   	\end{itemize}
                   \end{frame} 
                   
                   
                   \begin{frame}
                   	\frametitle{Natural spline  }
                   	\begin{itemize}
                   		\item The behavior of the cubic spline at boundary can be quite unstable.
                   		\item Natural spline is cubic spline but require the function to be linear
                   		on $(-\infty, \xi_1]$ and $[\xi_K, \infty)$. 
                   		\item With further restriction near boundary, natural spline regression
                   		generally behaves better than cubic spline regression.
                   		
                   		
                   	\end{itemize}
                   \end{frame} 
                  
               
                 \begin{frame}
                 	\frametitle{ }
                 	\begin{figure}
                 		\centering
                 		
                 		\centering
                 		\includegraphics[width=.7\linewidth]{ISLRFigures/7_4.pdf}
                 		%\caption{A subfigure}
                 		\caption{7.4. A cubic spline and a natural cubic spline, with three knots, fit to
                 			a subset of the Wage data. Natural spline has narrower confidence intervals near boundary
                 		}
                 	\end{figure}
                 \end{frame}
                 
                  
                  \begin{frame}
                  	\frametitle{Choice of number and locations of knots  }
                  	\begin{itemize}
                  		\item The behavior of the cubic spline at boundary can be quite unstable.
                  		\item Natural spline is cubic spline but require the function to be linear
                  		on $(-\infty, \xi_1]$ and $[\xi_K, \infty)$. 
                  		\item With further restriction near boundary, natural spline regression
                  		generally behaves better than cubic spline regression.
                  		\item Degree of freedom of natural spline with $K$ knots is $K+4-4=K$, but
                  		excluding the constant (absorbed in intercept), we usually call it $K-1$ degree of freedom.
                  		\item Example: natural cubic splines has $4=K-1$ degree of fredom corresponds to 
                  		$K=5$ knots and $K-2=3$ interior knots.
                  		
                  	\end{itemize}
                  \end{frame} 
                  
                   
                   \begin{frame}
                   	\frametitle{ }
                   	\begin{figure}
                   		\centering
                   		
                   		\centering
                   		\includegraphics[width=.8\linewidth]{ISLRFigures/7_5.pdf}
                   		%\caption{A subfigure}
                   		\caption{7.5. A natural cubic spline function with four degrees of freedom is
                   			fit to the Wage data. Left: A spline is fit to wage (in thousands of dollars) as
                   			a function of age. Right: Logistic regression is used to model the binary event
                   			wage$>250$ as a function of age. The fitted posterior probability of wage exceeding
                   			\$250,000 is shown.
                   		}
                   	\end{figure}
                   \end{frame}
                  
                    
                    \begin{frame}
                    	\frametitle{Choice of number and locations of knots  }
                    	\begin{itemize}
                    		\item Usually choose equally spaced knots within the range of values of inputs.
                    		\item If we know a function is highly varying somewhere, place more knots there, so that
                    		the spline function is also highly varying in the area.
                    		\item Try several choices of the number of knots, and use validation/cross-validation approach to determine the best.
                    		\item Many statistics software provide automatic choice of number and location of knots.
                    		
                    	\end{itemize}
                    \end{frame} 
                 
                   
                   
                     \begin{frame}
                     	\frametitle{Example: Wage data}
                     	\begin{figure}
                     		\centering
                     		
                     		\centering
                     		\includegraphics[width=.8\linewidth]{ISLRFigures/7_6.pdf}
                     		%\caption{A subfigure}
                     		\caption{Ten-fold cross-validated mean squared errors for selecting the
                     			degrees of freedom when fitting splines to the Wage data. The response is wage
                     			and the predictor age. Left: A natural cubic spline. Right: A cubic spline.
                     			It seems that three degrees of freedom for the natural
                     			spline and four degrees of freedom for the cubic spline are quite adequate
                     		}
                     	\end{figure}
                     \end{frame}
                     
                     
                     
                     \begin{frame}
                     	\frametitle{Comparison with Polynomial Regression}
                     	\begin{itemize}
                     		\item Regression splines often give superior results to polynomial regression.
                     		\item Splines introduce
                     		flexibility by increasing the number of knots but keeping the degree
                     		fixed.
                     		\item Polynomial increase model flexibity by increased order of power function, which 
                     		can be dangerously inapproximate for moderately large or small $X$ in absolute value.
                     		\item Polynomial function has poor boundary behavior.
                     		\item Natural spline is much better. 
                     		
                     	\end{itemize}
                     \end{frame} 
                     
                       \begin{frame}
                       	\frametitle{}
                       	\begin{figure}
                       		\centering
                       		
                       		\centering
                       		\includegraphics[width=.6\linewidth]{ISLRFigures/7_7.pdf}
                       		%\caption{A subfigure}
                       		\caption{\scriptsize 7.7. On the Wage data set, a natural cubic spline with 15 degrees
                       			of freedom is compared to a degree-15 polynomial. Polynomials can show wild
                       			behavior, especially near the tails.
                       		}
                       	\end{figure}
                       \end{frame}
\section{Smoothing spline}

\begin{frame}
	\frametitle{Smoothing spline}
	\begin{itemize}
		\item With to minimize
		$$ \sum_{i=1}^n (y_i - f(x_i))^2$$ 
		subject to certain smoothness constraints on $f(\cdot)$.
		\item  The most common constraint is $\ddot{f}$, the second derivative do not vary much. 
	
		\item  A natural choice is: minimizng 
		$$ \sum_{i=1}^n (y_i - f(x_i))^2\quad \hbox{subject to} \int \ddot f(x)^2 dx < s$$ 
	 
	
	\end{itemize}
\end{frame} 


\begin{frame}
	\frametitle{Smoothing spline}
	\begin{itemize}
			\item This is equivalent to 
			$$ \sum_{i=1}^n (y_i - f(x_i))^2 +\lambda\int \ddot f(x)^2 dx \eqno(7.11)$$ 
			where $\lambda$ is the tuning parameter.
			\item The first term is loss; the second term is roughness penalty.
			\item The function $f$ minimizing the above is called {\it smoothing spline.}
			
		\item The function that minimize that loss+roughness penalty is 
		a natural cubic spline with knots $x_1, ..., x_n$. 
	 
	\end{itemize}
\end{frame} 


\begin{frame}
	\frametitle{The tuning parameter}
	\begin{itemize}
		\item  $\lambda$ controls the amount of roughness penalty
		\item $\lambda= 0$: no penalty,   degree of freedom $= n$; overfit.
		 $$\hat f(x_i) = y_i$$ 
		\item $\lambda=\infty$: infinity penalty; $f$ must be linear, degree of freedom $=2$.
		$$\hat f(x) = \hat \beta_0 + \hat \beta_1 x_i, \quad \hbox{the least squares estimate}$$
		\item What the degree of freedom when $\lambda >0 $ and is finite?
		\item We call it {\it  effective degree of freedom}, denoted as $df_\lambda$.  
	 
		
	\end{itemize}
\end{frame} 


\begin{frame}
	\frametitle{Effective degree of freedom}
	\begin{itemize}
		\item  The $df_\lambda$   is a measure of the flexibility of the
		smoothing spline—the higher it is, the more flexible (and the lower-bias but
		higher-variance) the smoothing spline
		\item  Minimizing (7.11), let the fitted values be
		$$\hat {\bf y} = {\bf S}_\lambda {\bf y} \eqno(7.12)$$
		where $\hat {\bf y}  = (y_1, ... , y_n)^T$ is an $n$-vector, representing the fitted values
		at $x_1, ..., x_n$; and the sensitivity matrix ${\bf S}_\lambda$ is an $n\times n$ matrix, depending only on covariates.
		\item (It can be shown that the fitted values are  linear functions of {\bf y}).
		\item Then, the effective degree of freedom is 
		$$df_\lambda = trace ({\bf S})$$
	 
		
		
	\end{itemize}
\end{frame} 


\begin{frame}
	\frametitle{Choice of $\lambda$}
	\begin{itemize}
		\item  By cross validation.
		\item For leave-one-out cross-validation (LOOCV), it can be  shown
		$$\hbox{RSS}_{cv}(\lambda)= \sum_{i=1}^n (y_i - \hat f_\lambda^{(-i)}(x_i))^2
	= \sum_{i=1}^n \Bigl[ \frac{y_i - \hat f_{\lambda}(x_i) }{1- s_{\lambda, ii}}\Bigr]^2 $$
		where $s_{\lambda, ii}$ is the $i$-th diagonal element of ${\bf S}_\lambda$.
		\item One fit does it all! 
		\item Recall that this is the same as linear regression. 
		In fact, 
		$${\bf S}_{\infty} =  {\bf H}$$
		where ${\bf H}= {\bf X}({\bf X}^T {\bf X})^{-1} {\bf X}^T$ is the hat matrix in linear regression. 
		
	\end{itemize}
\end{frame} 

\begin{frame}
\frametitle{Fast computation of cross-validation I}

\begin{itemize}\small
  \item The leave-one-out cross-validation statistic is given by $$ CV = \frac{1}{N}\sum^N_{i=1} e^2_{[i]},$$
  where $e_{[i]}= y_i -\hat{y}_{[i]}$, the observations are given by $y_1,\dots, y_N$, and $\hat{y}_{[i]}$ is the predicted value obtained when the model is estimated with the $i$th case deleted.
  \item Suppose we have a linear regression model $\mathbf{Y} = \bfX \bfbeta + \mathbf{e}$. The $\hat{\bfbeta} = (\bfX^T \bfX)^{-1}\bfX^T \bfY$ and $\mathbf{H} = \bfX(\bfX^T \bfX)^{-1}\bfX^T$ is the hat matrix. It has this name because it is used to compute $\hat{\bfY}=\bfX \hat{\bfbeta} = \mathbf{H} \bfY$. If the diagonal values of $\mathbf{H}$ are denoted by $h_1,\dots,h_N$, then the leave-one-oout cross-validation statistic can be computed using
  $$CV=\frac{1}{N}\sum^N_{i=1} [e_i /(1-h_i)]^2,$$
  where $e_i = y_i - \hat{y}_i$ is predicted value obtained when the model is estimated with all data included.
\end{itemize}

\end{frame}

\begin{frame}
\frametitle{Fast computation of cross-validation II}

\textbf{Proof}
\begin{itemize}\small
  \item Let $\bfX_{[i]}$ and $\bfY_{[i]}$ be similar to $\bfX$ and $\bfY$ but with the 
$i$th row deleted in each case. Let $\bfx^T_i$ be the $i$th row of $\bfX$ and let $$\hat{\bfbeta}_{[i]} = (\bfX^T_{[i]}\bfX_{[i]})^{-1} \bfX^T_{[i]} \bfY_{[i]}$$
be the estimate of $\bfbeta$ without the $i$th case. Then $e_{[i]} = y_i - \bfx^T_i \hat{\bfbeta}_{[i]}$.
\item Now $\bfX^T_{[i]} \bfX_{[i]} = (\bfX^T \bfX -\bfx_i \bfx^T_i)$ and $\bfx^T_i (\bfX^T \bfX)^{-1} \bfx_i = h_i$. So by the Sherman-Morrison-Woodbury formula,
$$(\bfX^T_{[i]} \bfX_{[i]})^{-1}  = (\bfX^T \bfX)^{-1} + \frac{(\bfX^T \bfX)^{-1} \bfx_i \bfx^T_i (\bfX^T \bfX)^{-1}}{1-h_i}.$$

\end{itemize}

\end{frame}

\begin{frame}
\frametitle{Fast computation of cross-validation III}

\textbf{Proof}
\begin{itemize}\small
  \item Also note that $\bfX^T_{[i]}\bfY_{[i]} = \bfX^T\bfY - \bfx y_i$. Therefore
  \begin{align*}
  \hat{\bfbeta}_{[i]} &= \left[(\bfX^T \bfX)^{-1} + \frac{(\bfX^T \bfX)^{-1} \bfx_i \bfx^T_i (\bfX^T \bfX)^{-1}}{1-h_i}\right](\bfX^T \bfY - \bfx_i y_i)\\
  &=\hat\bfbeta - \left[\frac{(\bfX^T\bfX)^{-1}\bfx_i}{1-h_i}\right][y_i (1-h_i)-\bfx^T_i \hat\bfbeta + h_i y_i]\\
  &=\hat\bfbeta - (\bfX^T\bfX)^{-1} \bfx_i e_i /(1-h_i)
  \end{align*}
  \item Thus
  \begin{align*}
  e_{[i]} &= y_i - \bfx^T_i \hat{\bfbeta}_{[i]}\\
   &=y_i - \bfx^T_i\left[\hat{\bfbeta} - (\bfX^T\bfX)^{-1} \bfx_i e_i /(1-h_i)\right]\\
   &=e_i + h_i e_i/(1-h_i) = e_i/(1-h_i)
  \end{align*}
\end{itemize}

\end{frame}


                         \begin{frame}
                         	\frametitle{ }
                         	\begin{figure}
                         		\centering
                         		
                         		\centering
                         		\includegraphics[width=.7\linewidth]{ISLRFigures/7_8.pdf}
                         		%\caption{A subfigure}
                         		\caption{\scriptsize 7.8. Smoothing spline fits to the Wage data. The red curve results
                         			from specifying 16 effective degrees of freedom. For the blue curve, $\lambda$ was found
                         			automatically by leave-one-out cross-validation, which resulted in 6.8 effective
                         			degrees of freedom.
                         		}
                         	\end{figure}
                         \end{frame}
\section{Local regression}


\begin{frame}
	\frametitle{Local view }
	\begin{itemize}
		\item  Rather than considering fitting a function $f$ to the data, we just focus on 
		a target point, say $x_0$, and try to estimate $f(x_0) = \beta_0 $. 
		\item  Consider a weight function, often called kernel function, 
	 $k(t)$ which is nonnegative symmetric and becomes small when $|t|$ is large.
	 	\end{itemize}
	 \end{frame} 
	
	\begin{frame}
		\frametitle{Typical choice of kernels }
		\begin{itemize}
	  
	 	\item Uniform kernel:  $k(t) = 1/2 I(|t| \leq 1)$.
	 	\item Triangle kernel: $k(t) = (1- |t|)I(1t| \leq 1)$.
	 	\item Gaussian kernel:
	 	$k(t) = e^{-t^2/2}/\sqrt{2\pi}$
	 	\item Epanecknikov kernel:
	 	$k(t) = 3/4 (1- t^2)_+$
	 \item Logistic kernel: $k(t) = 1/(e^t + e^{-t} +2)$.
	 \item Sigmoid kernel: $k(t) = 2/(\pi(e^t + e^{-t}))$. 
		 
	\end{itemize}
\end{frame} 
  
 
 \begin{frame}
 	\frametitle{Local view }
 	\begin{itemize}
 		
 	\item use the kernel function to create {\it  weights } on each observation so that 
 	those with $x_i$ closer to $x_0$ gets more weights:
 	$$K_{i0} = \frac{1}{h} k(\frac{x_i -x_0}{h}) $$ 
 \item These weights create the ``Localness" surrounding $x_0$. $h$ is the bandwidth that is usually small.
 \item we can consider minimization
 $$ \sum_{i=1}^n K_{i0} (y_i - \beta_0 - \beta_1(x_i-x_0))^2$$
 Then, $\hat \beta_0$ is the estimator of $f(x_0)$. 
 \item This estimator is local linear estimator, since locally around $x_0$, we used linear function
 to approximate $f(x)$. 
 \item One can certainly consider local polynomial estimation, by considering local polynomial approximation.
 	 
 	\end{itemize}
 \end{frame} 
  

\begin{frame}
	\frametitle{Remark. }
	\begin{itemize}
		
		\item  Local linear estimate is also a linear function of ${\bf y}$, and there has expression of the form of (7.12). 
		\item The degree of freedom controled by the bandwidth. 
		\item Small  bandwidth results in small bias but high variance (and high effective degree of freedom).
		\item Can be difficult to implement with high dimension data, by the curse of dimensionality.
	\end{itemize}
\end{frame}
                         
                           \begin{frame}
                           	%\frametitle{Local regression}
                           	\begin{figure}
                           		\centering
                           		
                           		\centering
                           		\includegraphics[width=.8\linewidth]{ISLRFigures/7_9.pdf}
                           		%\caption{A subfigure}
                           		\caption{\scriptsize 7.9. Local regression illustrated on some simulated data, where the
                           			blue curve represents $f(x)$ from which the data were generated, and the light
                           			orange curve corresponds to the local regression estimate $\hat f(x)$. The orange colored
                           			points are local to the target point $x_0$, represented by the orange vertical line.
                           			The yellow bell-shape superimposed on the plot indicates weights assigned to each
                           			point, decreasing to zero with distance from the target point. The fit$\hat f(x_0)$ at 
                           			$x_0$ is
                           			obtained by fitting a weighted linear regression (orange line segment), and using
                           			the fitted value at $x_0$ (orange solid dot) as the estimate $\hat f(x)$.
                           		}
                           	\end{figure}
                           \end{frame}
\section{Generalized additive model}

\begin{frame}
	\frametitle{  }
	\begin{itemize}
		
		\item  With $p$ inputs, the general model should be
		$$ y_i = f(x_{i1}, ..., x_{ip}) + \epsilon_i.$$  
		\item  Difficult to model multivariate nonlinear function.
		\item  Restrict  search space to 
		$$\{ f(x_1, ..., x_p): f_1(x_1) + f_2(x_2) + ... f_p(x_p) \}$$
		\item The multivariate function is simple sum of nonlinear function of each varible. 
		\item This leads to the generalized additive model (GAM).
	\end{itemize}
\end{frame}


\begin{frame}
	\frametitle{The GAM  }
	\begin{itemize}
		
		\item  The model:
		$$ y_i = f_1(x_{i1}) + f_2(x_{i2}) ...+ f_p(x_{ip}) + \epsilon_i$$ 
		\item  The statistical estimation of $f_1$, ..., $f_p$ can be solved by
		taking advantage of 
		\item  1. the methodologies for nonlinear model for single input case.
		\item  2. a backfit algorithm.
	\end{itemize}
\end{frame}

%\begin{frame}
%	\frametitle{The GAM  }
%	\begin{itemize}
%		
%		\item  The model:
%		$$ y_i = f_1(x_{i1}) + f_2(x_{i2}) ...+ f_p(x_{ip}) + \epsilon_i$$ 
%		\item  The statistical estimation of $f_1$, ..., $f_p$ can be solved by
%		taking advantage of 
%		\item  1. the methodologies for nonlinear model for single input case.
%		\item 2.  a backfit algorithm.
%	\end{itemize}
%\end{frame}
% 
 
  

\begin{frame}
	\frametitle{The backfitting algorithm  }
	\begin{itemize}
		
		\item  Initialize the estimator of $f_1$, ..., $f_p$, denoted as
		$\hat f_1, ..., \hat f_p $.
		\item  Given estimates $\hat f_1, .., \hat f_{k-1}$, $\hat f_{k+1}$, ..., $ \hat f_p$,  
		compute 
		 $$\tilde y_i = y_i - \hat f_1(x_{i1}) - \hat f_{k-1}(x_{i, k-1})- \hat f_{k+1} (x_{i, k+1}) - ...- f_p(x_{ip})$$
		\item  Run nonlinear regression with response $\tilde y_i$ and single input $x_{ik}$, to obtain the estimate 
		of $f_k$. Update $\hat f_k$ by this estimate.
		\item Continue with the update of $f_{k+1}$. (If $k=p$ continue the update of $f_1$.)
		\item Repeat till convergence.
	\end{itemize}
\end{frame}


                             \begin{frame}
                             	\frametitle{Example: Wage data}
                             	\begin{figure}
                             		\centering
                             		
                             		\centering
                             		\includegraphics[width=.8\linewidth]{ISLRFigures/7_11.pdf}
                             		%\caption{A subfigure}
                             		\caption{7.11. For the Wage data, plots of the relationship between each feature
                             			and the response, wage, in the fitted model (7.16). Each plot displays the fitted
                             			function and pointwise standard errors. The first two functions are natural splines
                             			in year and age, with four and five degrees of freedom, respectively. The third
                             			function is a step function, fit to the qualitative variable education.
                             		}
                             	\end{figure}
                             \end{frame}
                               \begin{frame}
                               	\frametitle{Example: Credit data}
                               	\begin{figure}
                               		\centering
                               		
                               		\centering
                               		\includegraphics[width=.8\linewidth]{ISLRFigures/7_12.pdf}
                               		%\caption{A subfigure}
                               		\caption{7.12. Details are as in Figure 7.11, but now $f_1$  and $f_2$ are smoothing
                               			splines with four and five degrees of freedom, respectively.
                               		}
                               	\end{figure}
                               \end{frame}
                               
 
 \begin{frame}
 	\frametitle{Pros and Cons of GAM}
 	\begin{itemize}
 		
 		\item  It is nonlinear (potentially more accurate than linear if linear relation is not true)
 		\item  Additivity:
		\begin{itemize}
		\item examine the effect of each $x_j$ on the response $y$ while holding all of the other variables
 		fixed; 
		\item inference is possible; 
		\item the smoothness of the function $f_j$ for the variable $X_j$ can be summarized via degrees of freedom. 
		\end{itemize} 
 		\item  Interactions are missed: add low-dimensional interaction functions of the form $f_{jk}(X_j,X_k)$. 
 	\end{itemize}
 \end{frame}        
 
 
 
 \begin{frame}
 	\frametitle{ GAM also work for generalized linear model}
 	\begin{itemize}
 		
 		\item In general we have 
 	$$E(Y|X) = g(f_1(X_1) + ...+ f_p(X_p))$$
 		where $g$ is known link function. 
 		\item   For example, for logistic GAM:
 		$$P(Y=1|X) = \frac{\exp(f_1(X_1) +...+f_p(X_p))}{1 + \exp(f_1(X_1) +...+f_p(X_p))}$$
 	\end{itemize}
 \end{frame}    
 
 
 
 
              
                                 \begin{frame}
                                 	\frametitle{Logistic GAM}
                                 	\begin{figure}
                                 		\centering
                                 		
                                 		\centering
                                 		\includegraphics[width=.8\linewidth]{ISLRFigures/7_13.pdf}
                                 		%\caption{A subfigure}
                                 		\caption{7.13. For the Wage data, the logistic regression GAM given in (7.19)
                                 			is fit to the binary response I(wage$>250$)
                                 		Each plot displays the fitted function
                                 			and pointwise standard errors. The first function is linear in year, the second
                                 			function a smoothing spline with five degrees of freedom in age, and the third a
                                 			step function for education. There are very wide standard errors for the first
                                 			level $<$HS of education.
                                 		}
                                 	\end{figure}
                                 \end{frame}
                                  
                 
                 
                         
                                
                      
      		\begin{frame}
      			\frametitle{Exercises  }
      			
      			
      			{\sl  Run the R-Lab codes in Section 7.8 of ISLR
      				
      				Exercises 1, 2, 3, 5, 6, 7, 10 of Section 7.9 of ISLR (print 7) %1-7 of Section 7.9 of ISLR 
      			}
      			
      			
      			
      			
      			
      		\end{frame}
      		
      			\begin{frame}
      				\frametitle{  }
      				
      				
      				
      				End of Chapter 7. 
      				
      				
      			\end{frame}
      
    \end{document}
    
    
 