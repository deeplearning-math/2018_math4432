\documentclass{beamer}

\mode<presentation> {
	\usetheme{Pittsburgh}
}
\usepackage{mathtools}
\usepackage{mathrsfs}
\usepackage{mathtools}
\usepackage{graphicx}
\graphicspath{{./graphics/}}
\usepackage{booktabs}
\usepackage{picture}
\usepackage{algorithm}
\usepackage{algorithmic}
\usepackage{animate}
\usepackage{amsmath}
\usepackage{tikz}
\usepackage{tikz-qtree}
\usepackage{adjustbox}
\setbeamercovered{transparent}
\DeclarePairedDelimiter\abs{\lvert}{\rvert}
\hypersetup{
	colorlinks,
	linkcolor={red!50!black},
	citecolor={blue!50!black},
	urlcolor={blue!80!black}
}
\def\calA{{\cal A}}
\def\calF{{\cal F}}
\def\calP{{\cal P}}
\def\calE{{\cal E}}
\def\var{{\rm var}}

\def\bfA{{\bf A}}
\def\bfB{{\bf B}}
\def\bfC{{\bf C}}
\def\bfD{{\bf D}}
\def\bfE{{\bf E}}
\def\bfF{{\bf F}}
\def\bfG{{\bf G}}
\def\bfU{{\bf U}}
\def\bfV{{\bf V}}
\def\bfW{{\bf W}}
\def\bfX{{\bf X}}
\def\bfY{{\bf Y}}
\def\bfZ{{\bf Z}}

\def\bfa{{\bf a}}
\def\bfb{{\bf b}}
\def\bfc{{\bf c}}
\def\bfd{{\bf d}}
\def\bfe{{\bf e}}
\def\bff{{\bf f}}
\def\bfg{{\bf g}}
\def\bfu{{\bf u}}
\def\bfv{{\bf v}}
\def\bfw{{\bf w}}
\def\bfx{{\bf x}}
\def\bfy{{\bf y}}
\def\bfz{{\bf z}}

\setbeamertemplate{footline}{\insertframenumber}

\usetheme{Montpellier}  
\useoutertheme{infolines}
\usefonttheme{serif}


\title[Chapter 9]{Support Vector Machines} 

\author{ } 
\institute[ ]
{
	Chapter 9 \\ 
	\medskip
	\textit{ } 
}
%\date{\today}

\begin{document}
	 	
	 	\begin{frame}
	 		\titlepage % Print the title page as the first slide
	 	\end{frame}
	
 \begin{frame}
 	\frametitle{ }
 	\tableofcontents
 \end{frame}
 %###################################################################
 %###################################################################
     
      
      
      \begin{frame}
      	\frametitle{About this chapter}
      	\begin{itemize}
      		\item  Support vector machine is one of the most popular machine learning methodologies.
      		\item  Empirically successful, with well developed theory.
      		\item  One of the best off-the-shelf methods.
      		\item  We mainly address classification.
      	 
        	\end{itemize}
        \end{frame}
        
        	
        
\section{9.1. Maximal margin classifier}    
    
       \begin{frame}
       	\frametitle{Hyperplane in $R^p$}
       	\begin{itemize}
       		\item   ${\bf x} \in R^p$ (p-dimensional real space)
       		with components ${\bf x} = (x_1, ..., x_p)^T$.
       		\item Consider all ${\bf x}$ satisfying 
   %    		$$ f({\bf x}) = {\bf b}^T {\bf x} - c= \left< {\bf b}, {\bf x} \right> -c = b_1 x_1 + ... + b_p x_p -c = 0.$$
   		$$ f({\bf x}) = \beta_0 + \beta^T {\bf x} = \left< \beta, {\bf x} \right> = \beta_0 + \beta_1 x_1 + ... + \beta_p x_p = 0.$$
     		where $\beta=(\beta_1,\ldots,\beta_p)^T$. 
      
       		All such $\bf x$ defines a hyperplane: All ${\bf x}$ such that its projection on  $\beta$
       		is %$c{\bf b}/\|{\bf b}\|^2 = \left< {\bf b}/\|{\bf b}\|, {\bf x} \right>  {\bf b}/\|{\bf b}\|$.
		$$\left<  \frac{\beta}{\|{\beta}\|}, {\bf x} \right>  \frac{\beta}{\|{\beta}\|}=-\beta_0 \frac{\beta}{\|{\beta}\|^2} .$$
       	\end{itemize}
       	\end{frame}
       	
       	  \begin{frame}
       	  	\frametitle{Hyperplane in $R^p$}
       	  	\begin{itemize}
       		
       	    
       	  \item  $$ f( {\bf x})=0  \iff   {\bf x}= -\beta_0 \frac{\beta}{\|{\beta}\|^2} + y \qquad \hbox{  with  }  y \bot {\beta}.$$
	  %$$ f( {\bf x})=0  \iff   {\bf x}= c {\bf b}/ \|\bfb\|_2 + y \qquad \hbox{  with  }  y \bot {\bf b}.$$
       	  \item  $$ f( {\bf x}) > 0  \iff   {\bf x}= \tilde  \beta_0 \frac{\beta}{\|{\beta}\|^2} + y \qquad \hbox{  with  } y \bot {\beta},  \,\,  \tilde \beta_0 > -\beta_0 $$
	  %$$ f( {\bf x}) > 0  \iff   {\bf x}= \tilde c {\bf b}/ \|\bfb\|^2 + y \qquad \hbox{  with  } y \bot {\bf b},  \,\,  \tilde c > c $$
       	  \item   $$ f( {\bf x}) < 0  \iff   {\bf x}= \tilde \beta_0 \frac{\beta}{\|{\beta}\|^2} + y  \qquad \hbox{  with  } y \bot {\beta},  \,\, \tilde \beta_0 < -\beta_0 $$
	  here $\tilde \beta_0 = \left< {\bf x}  , {\beta} \right> $ and $f(\bfx)= \tilde{\beta}_0-(-\beta_0)$.
	  %$$ f( {\bf x}) < 0  \iff   {\bf x}= \tilde c {\bf b}/ \|\bfb\|^2 + y  \qquad \hbox{  with  } y \bot {\bf b},  \,\, 
       	  %\tilde c < c $$
       	    %here $\tilde c = \left< {\bf x}  , {\bf b} \right> $ and $f(\bfx)= \tilde c-c$.
       		        		 		
       		 	\end{itemize}
       		 \end{frame} 
       		
       	
      
      
       
            \begin{frame}
            	\frametitle{ }
            	\begin{itemize}
            		\item  $f({\bf x}) > 0 \iff {\bf x} $ is on one side of the hyperplane (at the same direction as ${\beta}$.)
            	
            	$f({\bf x}) <  0 \iff {\bf x} $ is on the other side of the hyperplane (at the opposite direction as ${\beta}$.)
            	 
            		\item For any vector ${\bf z} \in R^p$,
            		the signed distance of a point ${\bf z} \in R^p$ to this hyperplane is
		         $$f({\bf z})/\|\bfb\| =  ( \left <{\bf z}, {\beta} \right > +\beta_0 )/\|\beta\| = 
            		   \left <{\bf z}, {\beta}/\|{\beta}\| \right> + \beta_0  /\|\beta\|.$$
%            		$$f({\bf z})/\|\bfb\| =  ( \left <{\bf z}, {\bf b} \right > -c )/\|\bfb\| = 
%            		   \left <{\bf z}, {\bf b}/\|{\bf b}\| \right> - c  /\|\bfb\|.$$
            		\item If $\|\beta \|=1$, $f({\bf z})$ is the signed distance of ${\bf z}$ to the hyperplane 
            		defined by $f({\bfx})=0$.
            		
            	
            	\end{itemize}
            \end{frame} 
            
    
    \begin{frame}
    	\frametitle{ }
    	\begin{figure}
    		\centering
    		\includegraphics[width=.6\linewidth]{ISLRFigures/9_1.pdf}
    		%\caption{A subfigure}
    		\caption{\scriptsize 9.1. The hyperplane $1 + 2X_1 + 3X_2 = 0$ is shown. The blue region is
    			the set of points for which $1+2X_1 +3X_2 > 0$, and the purple region is the set of
    			points for which $1 + 2X_1 +3X_2 < 0$.
    		}
    	\end{figure}
    \end{frame}
  
  \begin{frame}
  	\frametitle{ Separating hyperplane}
  	\begin{itemize}
  		\item   Training Data:$(y_i, \bfx_i), i=1,..., n$,  with input $\bfx_i = (x_{i1}, ..., x_{ip})$ and 
  		two-class output
  		$y_i = \pm 1$.
  		
  	  \item Suppose the two classes are seperated by one hyperplane
  	  $$f(\bfx ) = \beta_0 + \beta_1 x_1 + ... + \beta_p x_p,$$
  	  meaning that, for all $i$ in one class $y_i = 1$, 
  	  $$ f(\bfx_i ) > 0,  \quad \hbox{ one side of the hyperplane;}$$
  	  and for all $i$ in the other class $y_i = -1$,
  	  $$ f(\bfx_i) < 0,  \quad \hbox{ the other  side of the hyperplane;}$$
  	  \item It can be equivalently expressed as
  	  $$ y_i f(\bfx_i) > 0, \quad \hbox{ for all $i=1,.., n$}$$
  	  \end{itemize}
  	   \end{frame}
  	   
  	   \begin{frame}
  	   	\frametitle{ }
  	   	\begin{itemize}
  	  
  	  \item If such separating hyperplane exists, it can be our classification rule:
  	  
  	  For any new/old observation with
  	   ${\bf x}^*$ such that $f(\bfx^*) > 0$, classify it as in the class $ + 1$.
  	  Otherwise, classify it as in class $-1$.
  	  
  	  \item Problem: If the two classes in training data
  	   are indeed separable by a hyperplane, which hyperplane is 
  	  the best?
  	
  \end{itemize}
\end{frame} 

  	 
 \begin{frame}
 	\frametitle{ }
 	\begin{figure}
 		\centering
 		\includegraphics[width=.7\linewidth]{ISLRFigures/9_2.pdf}
 		%\caption{A subfigure}
 		\caption{\scriptsize 9.2. Left: There are two classes of observations, shown in blue and
 			in purple, each of which has measurements on two variables. Three separating
 			hyperplanes, out of many possible, are shown in black. Right: A separating hyperplane
 			is shown in black. The blue and purple grid indicates the decision rule
 			made by a classifier based on this separating hyperplane: a test observation that
 			falls in the blue portion of the grid will be assigned to the blue class, and a test
 			observation that falls into the purple portion of the grid will be assigned to the
 			purple class.
 			
 		}
 	\end{figure}
 \end{frame}
 
           \begin{frame}
           	\frametitle{Maximal margin classifier }
           	\begin{itemize}
           		
           		\item  Maximal margin hyperplane: the separating hyperplane that
           		optimal
           		separating
           		hyperplane
           		is farthest from the training observations.
           		\item The seperating hyperplane such that the minimum
           		 distance of any training
           		point to the hyperplane is the largest.
           		
           		\item Creates a widest gap between the two classes.
           		
           		\item Points on the boundary hyperplane, those with smallest distance to 
           		the max margin   hyperplane, are called {\it support vectors}.
           		
           		They ``support" the maximal margin hyperplane in the sense vector
           		that if these points were moved slightly then the maximal margin hyperplane
           		would move as well
           		
           	\end{itemize}
           \end{frame} 
            
         
         \begin{frame}
         	\frametitle{ }
         	\begin{figure}
         		\centering
         		\includegraphics[width=.55\linewidth]{ISLRFigures/9_3.pdf}
         		%\caption{A subfigure}
         		\caption{\scriptsize 9.3. There are two classes of observations, shown in blue and in purple.
         			The maximal margin hyperplane is shown as a solid line. The margin is the
         			distance from the solid line to either of the dashed lines. The two blue points and
         			the purple point that lie on the dashed lines are the support vectors, and the distance
         			from those points to the hyperplane is indicated by arrows. The purple and
         			blue grid indicates the decision rule made by a classifier based on this separating
         			hyperplane.
         		}
         	\end{figure}
         \end{frame}
           
           \begin{frame}
           	\frametitle{ Computing the max margin hyperplane }
            
           	   
           		$$ \hbox{maximize}_{\beta_0, \beta_1, ..., \beta_p} M $$
           		$$ \hbox{subject to }  \sum_{j=1}^p \beta_j^2 = 1,  $$
           		$$ \hbox{and } y_i(\beta_0 + \beta_1 x_{i1} + ... + \beta_p x_{ip} ) \geq M \,\, 
           		\hbox{for all $i$ }$$
           		
           		
           		
           		
            
           \end{frame} 

             
             \begin{frame}
             	\frametitle{ }
             	\begin{itemize}
             		\item    Note that $M >0$ is the half of the width of the strip separating the two classes.
             		
             		\item  The eventual solution, the max margin hyperplane is determined by
             		the support vectors.  
             		
             		If $x_i$ on the correct side of the trip varies, the solution would 
             		remain same. 
             		
             		\item  The max margin hyperplane may vary a lot when the support vectors vary.
             		(high variance; see Figure 9.5)
             		
             		
             	\end{itemize}
             \end{frame} 
         
         
         \begin{frame}
         	\frametitle{ }
         	\begin{figure}
         		\centering
         		\includegraphics[width=.6\linewidth]{ISLRFigures/9_4.pdf}
         		%\caption{A subfigure}
         		\caption{\scriptsize 9.4. There are two classes of observations, shown in blue and in purple.
         			In this case, the two classes are not separable by a hyperplane, and so the
         			maximal margin classifier cannot be used.
         		}
         	\end{figure}
         \end{frame}
         
   \section{9.2. Support vector classifiers} 
   
   
   \begin{frame}
   	\frametitle{The non-seperable case }
   	\begin{itemize}
   		\item   In general, the two classes are usually not separable by any hyperplane.
   		
   		\item Even if they are, the max margin may not be desirable because of its high variance, and thus possible over-fit.
   		
   		\item  The generalization of the maximal margin classifier to the
   		non-separable case is known as the {\it  support vector classifier.}
   		
   		\item Use a soft-margin in place of the max margin. 
   		
   	\end{itemize}
   \end{frame}      
         
           
            	 
            		
             
         
         \begin{frame}
         	\frametitle{ }
         	\begin{figure}
         		\centering
         		\includegraphics[width=.7\linewidth]{ISLRFigures/9_5.pdf}
         		%\caption{A subfigure}
         		\caption{\scriptsize 9.5. Left: Two classes of observations are shown in blue and in
         			purple, along with the maximal margin hyperplane. Right: An additional blue
         			observation has been added, leading to a dramatic shift in the maximal margin
         			hyperplane shown as a solid line. The dashed line indicates the maximal margin
         			hyperplane that was obtained in the absence of this additional point.
         		}
         	\end{figure}
         \end{frame}
         
          \begin{frame}
          	\frametitle{  Non-perfect separation }
          	\begin{itemize}
          		\item   Consider a classifier based on a hyperplane
          		that does not perfectly separate the two classes, in the interest of
          		\begin{enumerate}
          			\item Greater robustness to individual observations, and
          			\item  Better classification of most of the training observations.
          		\end{enumerate}
          		
          		\item Soft-margin classifier (support vector classifier) allow some violation of
          		the margin: some can be on the wrong side of the margin (in the river) or even
          		wrong side of the hyperplane; see Figure 9.6.
          		
          	\end{itemize}
          \end{frame}    
         
         
         \begin{frame}
         	\frametitle{ }
         	\begin{figure}
         		\centering
         		\includegraphics[width=.8\linewidth]{ISLRFigures/9_6.pdf}
         		%\caption{A subfigure}
         		\caption{\scriptsize  9.6. next page
         		}
         	\end{figure}
         \end{frame}
         
         
         
           
              
              \begin{frame}
              
              		  FIGURE 9.6. Left: A support vector classifier was fit to a small data set. The
              		  hyperplane is shown as a solid line and the margins are shown as dashed lines.
              		  Purple observations: Observations 3, 4, 5, and 6 are on the correct side of the
              		  margin, observation 2 is on the margin, and observation 1 is on the wrong side of
              		  the margin. Blue observations: Observations 7 and 10 are on the correct side of
              		  the margin, observation 9 is on the margin, and observation 8 is on the wrong side
              		  of the margin. No observations are on the wrong side of the hyperplane. Right:
              		  Same as left panel with two additional points, 11 and 12. These two observations
              		  are on the wrong side of the hyperplane and the wrong side of the margin.
              		
              \end{frame}
              
                \begin{frame}
                	\frametitle{ Computing the support vector classifier   }
                    
                		$$ \hbox{maximize}_{\beta_0, \beta_1, ..., \beta_p} M $$
                		$$ \hbox{subject to } \qquad \qquad \sum_{j=1}^p \beta_j^2 = 1,  \qquad  \hbox{and } $$
                		$$ y_i(\beta_0 + \beta_1 x_{i1} + ... + \beta_p x_{ip} ) \geq M(1-\epsilon_i), \,\, \, \epsilon_i \geq 0 \,\, 
                		\hbox{for all $i$ }$$
                		$$ \hbox{and } \sum_{i=1}^n \epsilon_i \leq C,$$
                		
                		
                	  where $C$ is a nonnegative tuning parameter. $\epsilon_i$ are {\it slack variables}.
                		
                		 		
                		  
                		 \end{frame} 
                		 
                		  \begin{frame}
                		  	\frametitle{ The support vector classifier   }
                		   
                		  		
                		  The solution of this optimization is the support vector classsifier:
                		$$f({\bf x}) = \beta_0 + \beta_1 x_1 + ...+ \beta_p x_p.$$
                		And we classify an observation in $+ 1$ class if $f({\bf x}) > 0$; else into
                		$-1$ class. 	
                
                \end{frame} 
                
                
                \begin{frame}
                	\frametitle{ Understaning the slack variable $\epsilon_i$   }
                	\begin{itemize}
                		\item  $\epsilon_i = 0 \iff $ the $i$-th observation is on the correct side of the margin
                		
                		\item   $\epsilon_i > 0 \iff $ the $i$-th observation is on the wrong side of the margin
                		 \item $\epsilon_i >1   \iff $ the $i$-th observation is on the wrong side of the hyperplane.
                		
                	\end{itemize}
                \end{frame}   
                
                 
                 \begin{frame}
                 	\frametitle{ Understanding tuning parameter $C$ }
                 	\begin{itemize}
                 		\item $C$ is a $budget$ for the amount that the margin can be violated
                 		by the $n$ observations
                 		
                 		\item   $C=0 \iff$ no budget 
                 		       
                 		       As a result $\epsilon_i = 0$ for all $i$.
                 		       
                 		       The classifier is a maximal margin classifier, which exists only if the two classes are separable by hyperplanes.
                 		 \item Larger $C$, more tolerance of margin violation. 
                 		\item  No more than $C$ observations can be on the wrong side of 
                 		    the soft-margin classifier hyperplane.
                 		
                 		\item As $C$ increases, the margin widens and more violations of the margin.
                 	\end{itemize}
                 \end{frame}   
                 
                 \begin{frame}
                 	\frametitle{ Understanding tuning parameter $C$ }
                 	\begin{itemize}
                 		\item $C$ controls the bias-variance trade-off.
                 		
                 		\item  Small $C$ high variance, small bias. 
                 		
                 		\item Large $C$: small variance, high bias. 
                 		
                 		\item $C$ can be determined by using cross validation.
                 		
                 	 
                 	\end{itemize}
                 \end{frame}   
                 
                       
                       	       \begin{frame}
                       	       	\frametitle{ Support vectors}
                       	       	\begin{itemize}
                       	       		\item Observations
                       	       		that lie directly on the margin, or on the wrong side of the margin for
                       	       		their class, are known as support vectors.
                       	       		
                       	       		\item  Only the support vectors affect the 
                       	       		support vector classifier.
                       	       	 
                       	       		\item Those strictly on the correct side of the margin do  not.
                       	       		(robustness, analgous to median)
                       	       		\item Larger $C$ $\Longrightarrow$ more 
                       	       		violations, $\Longrightarrow$ more 
                       	       	support vectors, $\Longrightarrow$ smaller variance and
                       	       	more robust classfier.
                       	       		
                       	       		
                       	       	\end{itemize}
                       	       \end{frame}   
               
               \begin{frame}
               	\frametitle{ }
               	\begin{figure}
               		\centering
               		
               		\centering
               		\includegraphics[width=.7\linewidth]{ISLRFigures/9_7.pdf}
               		%\caption{A subfigure}
               		\caption{ 9.7.  
               		 }
               		 
               		 
               	\end{figure}
               \end{frame}
              
               
               \begin{frame}
                
             
               	 
               	 
               		%\caption{A subfigure}
               	 {Figure 9.7. A support vector classifier was fit using four different values of the
               			tuning parameter $C$ in (9.12) and ��(9.15). The largest value of $C$ was used in the top
               			left panel, and smaller values were used in the top right, bottom left, and bottom
               			right panels. When $C$ is large, then there is a high tolerance for observations being
               			on the wrong side of the margin, and so the margin will be large. As $C$ decreases,
               			the tolerance for observations being on the wrong side of the margin decreases,
               			and the margin narrows.
               		}
               		
               		
             
               \end{frame}
               
  \section{9.3. Support vector machines}
  
  
     
     \begin{frame}
     	\frametitle{ }
     	\begin{figure}
     		\centering
     		
     		\centering
     		\includegraphics[width=.7\linewidth]{ISLRFigures/9_8.pdf}
     		%\caption{A subfigure}
     		\caption{ 9.8. Nonlinear boundaries. Left: The observations fall into two classes, with a non-linear
     			boundary between them. Right: The support vector classifier seeks a linear boundary,
     			and consequently performs very poorly.}
     		
     		
     	\end{figure}
     \end{frame}
  
               
                   \begin{frame}
                   	\frametitle{ Extending to nonlinear boundary}
                   	\begin{itemize}
                   		\item In practice, we are sometimes faced with non-linear class boundaries
                   		
                   		\item   Linear classifier could perform poorly.
                   		
                   		\item  Need nonlinera classifier.
                   		
                   		\item  As in the extension to the polynomial regression from linear regression, we can consider {\it enlarge the feature space} from the original
                   		$p$ inputs to polynomials (of certain order) of the inputs. 
                   	 
                   		
                   	\end{itemize}
                   \end{frame}   
               
                   \begin{frame}
                   	\frametitle{ Extending to  quadratic inputs  }
                   	\begin{itemize}
                   		\item Rather than constructing the support vector classifier using $p$ features:
                   		$$X_1, ..., X_p.$$
                   		\item   we use $2p$ features:
                   			$$X_1, X_1^2, ..., X_p, X_p^2.$$
                   		
                   		\item  Treat them as $2p$ original inputs, and fit the support vector classifier.
                   		
                   		\item  The separating hyperplane is a hyperplane in $R^{2p}$, which should
                   		be a linear equation:
                   		$$\beta_0 + \beta_1X_1 + \beta_2 X_p+ \beta_{p+1} X_1^2 
                   		+ ... +\beta_{2p} X_p^2 =0$$
                   		
                   		\item This is a quadrtic equation in $X_1, ..., X_p$. Thus 
                   		the separating surface in $R^p$ in terms of $X_1, ..., X_p$ corresponds 
                   		to a quadratic surface in $R^p$.
              
                   	\end{itemize}
                   \end{frame}   
                   
                    \begin{frame}
                    	\frametitle{ Extending to polynomial inputs  }
                    	\begin{itemize}
                    		\item  Can extend to polynomial of any given order $d$. 
                    		
                    		\item   Could lead to too many features, too large feature space; thus overfit.  
                    		
                    		\item   Higher powers are unstable. 
                    		
                    	\end{itemize}
                    \end{frame}   
                   
            
            \begin{frame}
            	\frametitle{ Key observation}
            	\begin{itemize}
            		\item  The linear support vector classifier can be represented as
            	$$	f(\bfx ) = \beta_0 + \sum_{i=1}^n \alpha_i \left< {\bf x}_i , {\bf x}\right>$$
            	 
            	 
            	\item  $\alpha_i \not=0$ only for all  support vectors. 
            	
            	\item $\alpha_i$ can also be computed based on $\left<\bfx_j , \bfx_k\right>$.
            	
            	\item Only the inner product of the feature space is relevant in computing
            	the linear support vector classifier.	
            	\end{itemize}
            \end{frame}   
            
            
            
               \begin{frame}
               	\frametitle{ The kernel trick}
               	\begin{itemize}
               		\item  The inner product $\left< \cdot, \cdot\right>$ 
               		is a bivariate function (satisfying some property).
               		
               		
               		
               		\item  It can be genearlized to kernel functions
               		$$ K({\bfx} , {\bf z})$$
               		which is positive definite.
               		
               		\item  The classifier can be expressed
               		as 
               		$$ f(\bfx) = \beta_0 + \sum_{i=1}^n a_i K({\bfx}, {\bfx}_i)$$
               		
               		
               		  	\end{itemize}
               		  \end{frame} 
               		  
               		 \begin{frame}
               		 	\frametitle{ Examples of kernels}
               		 	\begin{itemize}
               		\item Examples of the kernel function are:
               	 
               			\item linear kernel 
               			$$K(x_i, x_j) = \left< x_i, x_j\right> = x_i^T x_j.$$
               			\item polynomial kernel of degree $d$:
               			$$K(x_i, x_j) = (1+  \left< x_i, x_j\right> )^d .$$ 
               		\item Gaussian radial kernel:
               		
              $$K(x_i, x_j) = \exp(-\gamma   \| x_j - x_j\|^2 ), \quad \gamma > 0 .$$ 
               		\item Only the inner product of the feature space is relevant in computing
               		the linaer support vector classfier.	
               		
               	\end{itemize}
               \end{frame}  
                \begin{frame}
                	\frametitle{ }
                	\begin{figure}
                		\centering
                		\includegraphics[width=.7\linewidth]{ISLRFigures/9_9.pdf}
                		%\caption{A subfigure}
                		\caption{\scriptsize 9.9. Left: An SVM with a polynomial kernel of degree 3 is applied to
                			the non-linear data from Figure 9.8, resulting in a far more appropriate decision
                			rule. Right: An SVM with a radial kernel is applied. In this example, either kernel
                			is capable of capturing the decision boundary.
                			In
                		}
                	\end{figure}
                \end{frame}
                
             \begin{frame}
             	\frametitle{ }
             	\begin{itemize}
             		\item The radial kernel has local behavior.
             		
             		\item To predict the class for a new observation with input $\bfx$, 
             		
             $$ f(\bfx) = \beta_0 + \sum_{i=1}^n a_i \exp(-\gamma   \| \bfx - {\bf x}_i\|^2 )$$ 
             		
             		\item A training data point ${\bf x}_i$ being far away from ${\bf x}$ will 
             		have little effect to the sign of $f(\bfx)$.
             	
             		
             	\end{itemize}
             \end{frame}  
            
                \begin{frame}
                	\frametitle{The enlarged feature space. }
                	\begin{itemize}
                		\item  The support vector machine actually enlarges the original 
                		feature space to a space of kernel functions.
                		$$ {\bf x_i} \to K(\cdot, {\bfx}_i).$$
                	
                		\item  The original space of $p$ inputs has dimension $p$.
                		\item The enlarged space of features, the function space, is infinite
                		dimension!
                		
                	    \item  In actual fitting of the support vector machine, we only need to 
                	    compute the $K(x_i, x_j)$ for all $x_i, x_j$ in training data. 
                	    
                	    \item Do not have to work with the enlarged feature space of infinite dimension.
                	\end{itemize}
                \end{frame}  
            
             \begin{frame}
             	\frametitle{Example: the Heart data. }
             	\begin{itemize}
          \item   In Chapter 8 we apply decision trees and related methods to the Heart data.
          \item   The aim is to use 13 predictors such as Age, Sex, and Chol in order to predict
            whether an individual has heart disease.
            
           \item  We now investigate how an SVM
            compares to LDA on this data.
            
            \item 297 subjects,   randomly split into 207 training and
            90 test observations.
            	\end{itemize}
            \end{frame}
              
             
              		 
             
                 
                 
                 	 
                 
                 
                   \begin{frame}
                   	\frametitle{ }
                   	\begin{figure}
                   		\centering
                   		\includegraphics[width=.7\linewidth]{ISLRFigures/9_10.pdf}
                   		%\caption{A subfigure}
                   		\caption{\scriptsize ROC curves for the Heart data training set. Left: The support
                   			vector classifier and LDA are compared. Right: The support vector classifier is
                   			compared to an SVM using a radial basis kernel with $\gamma=10^{-3},
                   			10^{-2}$ and $10^{-1}$.
                   			 
                   		}
                   	\end{figure}
                   \end{frame}
                   
                   
                     \begin{frame}
                     	\frametitle{ }
                     	\begin{figure}
                     		\centering
                     		\includegraphics[width=.7\linewidth]{ISLRFigures/9_11.pdf}
                     		%\caption{A subfigure}
                     		\caption{\scriptsize 9.11. ROC curves for the test set of the Heart data. Left: The support
                     			vector classifier and LDA are compared. Right: The support vector classifier is
                     			compared to an SVM using a radial basis kernel with $\gamma=10^{-3},
                     			10^{-2}$ and $10^{-1}$.
                     		}
                     	\end{figure}
                     \end{frame}
                     
  \section{Appendix: Primal-Dual support vector classifiers}

           \begin{frame}
           	\frametitle{ Appendix: Equivalent reformulation of Hard Margin}
            
           	   
           		$$ \hbox{maximize}_{\beta_0, \beta_1, ..., \beta_p} M $$
           		$$ \hbox{subject to }  \sum_{j=1}^p \beta_j^2 = 1,  $$
           		$$ \hbox{and } y_i(\beta_0 + \beta_1 x_{i1} + ... + \beta_p x_{ip} ) \geq M \,\, 
           		\hbox{for all $i$ }$$
           		$$\Leftrightarrow $$
          		$$ \hbox{minimize}_{\beta_0, \beta_1, ..., \beta_p} \|\beta\|^2 := \sum_j \beta_j^2 $$
           		$$ \hbox{subject to }  y_i(\beta_0 + \beta_1 x_{i1} + ... + \beta_p x_{ip} ) \geq1 \,\, 
           		\hbox{for all $i$ },$$
		using $M=1/\|\beta\|$.
           		
           \end{frame} 

   \begin{frame}
    	\frametitle{ }
    	\begin{figure}
    		\centering
    		\includegraphics[width=.6\linewidth]{ISLRFigures/hardmargin.pdf}
    		\caption{Separating hyperplane with margin}
    	\end{figure}
    \end{frame}


            \begin{frame}
           	\frametitle{ Appendix: Lagrangian }
            
           	   Lagrangian: for $\alpha_i\geq 0$,
         	$$ \max_{\alpha\geq 0} \min_{\beta} L(\beta,\alpha) := \frac{1}{2}\|\beta\|^2 - \sum_{i=1}^n \alpha_i (y_i(\beta_0 + \beta_1 x_{i1} + ... + \beta_p x_{ip}) -1 )$$ 		
		So 
		$$ 0=\frac{\partial L}{\partial \beta}\Rightarrow \hat\beta = \sum_i \alpha_i y_i \bfx_i $$
       \    	$$ 0=\frac{\partial L}{\partial \beta_0}\Rightarrow  \sum_i \hat\alpha_i y_i =0 $$
	and complementary condition
		$$  \hat\alpha_i (y_i (\hat\beta_0 +\left< \hat\beta, \bfx_i\right>)-1) =0,\ \ \  \hbox{for all $i$ } $$
           \end{frame} 

            \begin{frame}
           	\frametitle{ Appendix: Support Vectors }
            
           	Complementary condition $  \hat\alpha_i (y_i (\hat\beta_0 +\left< \hat\beta, \bfx_i\right>)-1) =0,\ \ \  \hbox{for all $i$ } $ implies that
	$$  y_i (\hat\beta_0 +\left< \hat\beta, \bfx_i\right>) > 1 \Rightarrow \hat\alpha_i =0 $$
	$$  y_i (\hat\beta_0 +\left< \hat\beta, \bfx_i\right> )= 1 \Rightarrow \hat\alpha_i \geq 0 $$
	
	Those sample point $i$'s such that $\hat\alpha_i>0$, are called \emph{support vectors} (sv), which decided the maximal margin hyperplane 
		$$ \hat\beta = \sum_{i\in sv} \hat\alpha_i y_i \bfx_i $$
           and $\hat\beta_0$ can be uniquely decided by any support vector $s$ using $\hat\beta_0 = y_s -  \left< \hat\beta,\bfx_s\right>$. 
           \end{frame} 

              \begin{frame}
           	\frametitle{ Appendix: Dual Formulation }
            
           	After plugging $\hat\beta$ in the Lagrangian, it gives 
	$$ \min_\alpha \frac{1}{2} \sum_{i,j=1}^n \alpha_i \alpha_j y_i y_j \bfx_i^T \bfx_j   $$
	subject to 
		$$  \alpha_i \geq 0 $$
		$$  \sum_i \alpha_i y_i =0 $$

           \end{frame} 
           
            \begin{frame}
           	\frametitle{ Appendix: Equivalent reformulation of Soft-margin Classifier }
            
           	   
           		$$ \hbox{maximize}_{\beta_0, \beta_1, ..., \beta_p} M $$
           		$$ \hbox{subject to }  \sum_{j=1}^p \beta_j^2 = 1,  $$
           		$$ \hbox{and } y_i(\beta_0 + \beta_1 x_{i1} + ... + \beta_p x_{ip} ) \geq M (1-\epsilon_i), \,\, \epsilon_i\geq 0, 
           		\hbox{   for all $i$ }$$
		$$ \hbox{and } \sum_{i=1}^n \epsilon \leq C, $$ 
           		$$\Leftrightarrow $$
          		$$ \hbox{minimize}_{\beta_0, \beta_1, ..., \beta_p} \|\beta\|^2 + C \sum_{i=1}^n \xi_i $$
           		$$ \hbox{subject to }  y_i(\beta_0 + \beta_1 x_{i1} + ... + \beta_p x_{ip} ) \geq 1 - \xi_i, \,\, \xi_i\geq 0 
           		\hbox{ for all $i$ },$$
%		using $M=1/\|\beta\|$.          		
           \end{frame} 

   \begin{frame}
    	\frametitle{ }
    	\begin{figure}
    		\centering
    		\includegraphics[width=.6\linewidth]{ISLRFigures/softmargin.pdf}
    		\caption{Separating hyperplane with margin}
    	\end{figure}
    \end{frame}


            \begin{frame}
           	\frametitle{ Appendix: The Lagrangian }
            
           	   Lagrangian: for $\alpha_i\geq 0$, $\mu_i\geq 0$, $\xi_i\geq 0$,
          	$$L(\beta,\xi,\alpha,\mu) = \frac{1}{2}\|\beta\|^2 +C \sum_{i=1}^n \xi_i - \sum_{i=1}^n \alpha_i [y_i(\beta_0 + \bfx_i^T \beta) -(1-\xi_i)] - \sum_{i=1}^n \mu_i \xi_i,$$
		So for all $i$,
		\begin{equation}  \label{eq:p1}
		0=\frac{\partial L}{\partial \beta}\Rightarrow \hat\beta = \sum_i \hat\alpha_i y_i \bfx_i 
		\end{equation}
         	\begin{equation} \label{eq:p2} 
		0=\frac{\partial L}{\partial \beta_0}\Rightarrow  \sum_i \hat\alpha_i y_i =0 
		\end{equation}
       		\begin{equation} \label{eq:p3} 
		0=\frac{\partial L}{\partial \xi_i}\Rightarrow  \hat\alpha_i + \mu_i = C
		\end{equation}
	and complementary condition
		$$  \hat\alpha_i [y_i (\hat\beta_0 +\bfx_i^T \hat\beta)-(1-\xi_i)] =0$$
		$$ \mu_i \xi_i = 0 $$
           \end{frame} 

           \begin{frame}
           	\frametitle{ Appendix: Support Vectors }
            
           	Complementary condition $  \hat\alpha_i [y_i (\hat\beta_0 +\bfx_i^T\hat\beta)-(1-\xi_i)] =0$, $\xi_i\geq 0$, $\Rightarrow$
	$$  y_i (\hat\beta_0 +\left< \hat\beta, \bfx_i\right>) > 1 \Rightarrow \hat\alpha_i =0 $$
	$$  y_i (\hat\beta_0 +\left< \hat\beta, \bfx_i\right> )= 1-\xi_i \Rightarrow C\geq \hat\alpha_i \geq 0 $$
	$$ \mu_i\xi_i =0 \Rightarrow \xi_i>0, \mu_i =0, \hat\alpha_i = C - \mu_i = C $$
	Those samples such that $C\geq \hat\alpha_i>0$, are called \emph{support vectors} (sv),
		$$ \hat\beta = \sum_{i\in sv} \hat\alpha_i y_i \bfx_i $$
            and those s.t. $\alpha_i = C$ are within margin (violators), $\xi_i=0$ are on margin (deciding $\hat\beta_0$). 
           \end{frame} 

                \begin{frame}
           	\frametitle{ Appendix: Dual Problem }
            
           	Substituting \eqref{eq:p1}-\eqref{eq:p3} into the Lagrangian, it gives the dual optimization problem
	$$ \min_\alpha \frac{1}{2} \sum_{i,j=1}^n \alpha_i \alpha_j y_i y_j \bfx_i^T \bfx_j  -\sum_{i=1}^n \alpha_i  $$
	subject to 
		$$  C\geq \alpha_i \geq 0 $$
		$$  \sum_i \alpha_i y_i =0 $$
One can replace $\bfx_i^T \bfx_j$ by $K(\bfx_i,\bfx_j)$ using the kernel trick for nonlinear SVMs. 
           \end{frame} 
                   
\section{9.4. SVMs with more than two classes}

     \begin{frame}
     	\frametitle{One-versus-one approach  }
     	\begin{itemize}
     		\item    With $K>2$ classes.
     		
     		\item Run a SVM on each of the ${K \choose 2}$ pairs of classes. 
     		
     		\item We obtain ${K\choose 2}$ SVMs.
     		
     		\item  For every test observations, compute the number of times it is classified 
     		into class $k$ by all the SVMs, denote as $d_k$.
     		
     		\item Classify it into the class with highest $d_k$ (majortiy vote).
     		
     		
     	\end{itemize}
     \end{frame}

\begin{frame}
	\frametitle{One-versus-all approach  }
	\begin{itemize}
		\item    With $K>2$ classes.
		
		\item Run a SVM on class $k$ (coded as $+ 1$) versus  class  ``not-$k$" (coded as
		$-1$): 
		$f_k({\bfx})$. (Note that the larger $f_k(\bfx)$, the more likely ${\bfx}$ is in class $k$.
		
		\item For a new test observation with {\bfx}, compute 
	 assign to the class with largest $f_k({\bfx})$. 
		
	
		
	\end{itemize}
\end{frame}

\section{9.5. Relationshiop to logistic regression}


\begin{frame}
	\frametitle{Recall the ``Loss + Penalty" formula  }
	\begin{itemize}
		\item   Minimize, for $f $ in certain space,
		$$\sum_{i=1}^n L(y_i, f({\bf x}_i)) + \lambda P(f) $$ 
		\item  Ridge regression: for linear $f$,
		
			$$\sum_{i=1}^n  (y_i-  f({\bf x}_i))^2 + \lambda \sum_{j=1}^p \beta_j^2 $$ 
		\item  Lasso regression: for linear $f$
			$$\sum_{i=1}^n  (y_i-  f({\bf x}_i))^2 + \lambda \sum_{j=1}^p |\beta_j| $$ 

		
	\end{itemize}
\end{frame}

\begin{frame}
	\frametitle{Logistic regression for classification  }
	\begin{itemize}
		\item Data: $(y_i, {\bf x}_i)$, $y_i = \pm 1$.
		\item The logistic model assumes 
		$$P(Y= 1|X) = 1/(1+e^{-f(X)}); \quad P(Y=-1|X) = 1/(1+ e^{f(X)})$$
		That is 
		$$ P(Y=y|X)= 1/(1+e^{-yf(X)})$$
		
				
			\end{itemize}
		\end{frame}
		
		\begin{frame}
			\frametitle{Logistic regression for classification  }
			\begin{itemize}
				
		\item   The negative of logistic likelihood (a.k.a. binomial deviance, cross-entropy)
		$$     \sum_{i=1}^n   \log(1+ e^{-y_if({\bfx_i}) } ) $$ 
		
		\item Note that in AdaBoost, exponential loss is used
		$$     \sum_{i=1}^n   e^{-y_if({\bfx_i}) } $$ 
						
	\end{itemize}
\end{frame}


		\begin{frame}
			\frametitle{Logistic regression with regularization  }
			\begin{itemize}
				
		\item Logistic loss with ridge $l_2$ penalty
		$$   \sum_{i=1}^n   \log(1+ e^{-y_if({\bfx_i}) } ) + \lambda \sum_{j=1}^p \beta_j^2 $$ 
		\item Logistic loss with Lasso $l_1$ penalty: 
		$$  \sum_{i=1}^n   \log (1+ e^{-y_if({\bfx_i}) } )+ \lambda \sum_{j=1}^p |\beta_j| $$ 
		\item AdaBoost uses \emph{coordinate descent} to solve $l_1$-penalty approximately
		$$  \sum_{i=1}^n   e^{-y_if({\bfx_i}) }+ \lambda \sum_{j=1}^p |\beta_j| $$ 				
	\end{itemize}
\end{frame}

		

\begin{frame}
	\frametitle{The SVM    }
	\begin{itemize}
		\item Data: $(y_i, {\bf x}_i)$, $y_i = \pm 1$.
		
		\item   SVM is a result of ``hige loss + ridge penalty":
		$$    \sum_{i=1}^n   \hbox{max} [0, 1-y_i f(x_i)] + \lambda \sum_{j=1}^p \beta_j^2.$$ 
		
	 where $f(\bfx)= \beta_0 + \beta_1 x_1 + ... + \beta x_p$. Note that $\xi_i = \max[0,1-y_i f(x_i)]\geq 0$. 
	 
	  \item the hinge loss function : $l(u)=(1-u)_+$.
	  \item the logistic loss function: $l(u)=\log(1+e^{-u})$.
	  \item the exponential loss function: $l(u)=e^{-u}$.		
		
	\end{itemize}
\end{frame}



                    
                       \begin{frame}
                       	\frametitle{ }
                       	\begin{figure}
                       		\centering
                       		\includegraphics[width=.5\linewidth]{ISLRFigures/9_12.pdf}
                       		%\caption{A subfigure}
                       		\caption{\scriptsize 9.12. The SVM and logistic regression loss functions are compared,
                       			as a function of 
                       			$y_i(\beta_0 + \beta_1 x_{i1} +... \beta_p x_{ip})$.
                       		 When $y_i(\beta_0 + \beta_1 x_{i1} +... \beta_p x_{ip})$ is
                       			greater than 1, then the SVM loss is zero, since this corresponds to an observation
                       			that is on the correct side of the margin. Overall, the two loss functions have quite
                       			similar behavior.
                       			 
                       		}
                       	\end{figure}
                       \end{frame}
                       
               
               
                   
                       \begin{frame}
                       	\frametitle{ }
                       	\begin{figure}
                       		\centering
                       		\includegraphics[width=.6\linewidth]{ISLRFigures/ESL_figures10_4.pdf}
                       		%\caption{A subfigure}
%                       		\caption{\scriptsize 9.12. The SVM and logistic regression loss functions are compared,
%                       			as a function of 
%                       			$y_i(\beta_0 + \beta_1 x_{i1} +... \beta_p x_{ip})$.
%                       		 When $y_i(\beta_0 + \beta_1 x_{i1} +... \beta_p x_{ip})$ is
%                       			greater than 1, then the SVM loss is zero, since this corresponds to an observation
%                       			that is on the correct side of the margin. Overall, the two loss functions have quite
%                       			similar behavior.
%                       			 
%                       		}
                       	\end{figure}
                       \end{frame}

                        \begin{frame}
                       	\frametitle{A Separable Two-Class Problem}
                       	\begin{figure}
                       		\centering
                       		\includegraphics[width=.6\linewidth]{ISLRFigures/separable.png}
                       		%\caption{A subfigure}
%                       		\caption{\scriptsize 9.12. The SVM and logistic regression loss functions are compared,
%                       			as a function of 
%                       			$y_i(\beta_0 + \beta_1 x_{i1} +... \beta_p x_{ip})$.
%                       		 When $y_i(\beta_0 + \beta_1 x_{i1} +... \beta_p x_{ip})$ is
%                       			greater than 1, then the SVM loss is zero, since this corresponds to an observation
%                       			that is on the correct side of the margin. Overall, the two loss functions have quite
%                       			similar behavior.
%                       			 
%                       		}
                       	\end{figure}
                       \end{frame}

                        \begin{frame}
                       	\frametitle{Logistic Regression has Max-Margin Solution Asymptotically at $\infty$}
                       	\begin{figure}
                       		\centering
                       		\includegraphics[width=.6\linewidth]{ISLRFigures/logisticloss.png}
                       		%\caption{A subfigure}
                       		\caption{\scriptsize Logistic loss surface: the line indicates the direction of max-margin solution of support vector machine for the separable two-class problem.                        			 
                       		}
                       	\end{figure}
                       \end{frame}

                        \begin{frame}
                       	\frametitle{Exponential Loss Minimization has Max-Margin Solution Asymptotically at $\infty$}
                       	\begin{figure}
                       		\centering
                       		\includegraphics[width=.5\linewidth]{ISLRFigures/exploss.png}
                       		%\caption{A subfigure}
                       		\caption{\scriptsize Exponential loss surface and the max-margin solution of SVM (line direction).                        			 
                       		}
                       	\end{figure}
                       \end{frame}
                  
      		\begin{frame}
      			\frametitle{Exercises  }
      			
      			
      			{\sl  Run the R-Lab codes in Section 9.6 of ISLR
      				
      				Exercises
      	 1-3  of Section  9.7 of ISLR 
      			}
      			
   
      		\end{frame}
      		
      			\begin{frame}
      				\frametitle{  }
      				
      				
      				
      				End of Chapter 9. 
      				
      				
      			\end{frame}
      
    \end{document}
    
    
 